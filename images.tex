\batchmode
\documentclass{book}
\RequirePackage{ifthen}




\usepackage[paperheight=8.5in,paperwidth=6in,margin=1in,heightrounded]{geometry}
\usepackage{verbatim}
\usepackage{fancyvrb}
\usepackage{hyperref}
\hypersetup{colorlinks=true}
\usepackage{glossaries}
\usepackage{endnotes}
\usepackage{parskip}
\usepackage{multicol}
\usepackage{csquotes}%
\providecommand{\squeezeup}{\vspace{-3cm}}     
\usepackage{makeidx}
\makeindex
\usepackage{graphicx}
\usepackage{listings}
\usepackage{xcolor}
\usepackage{xparse}
\usepackage{lineno}
\usepackage{afterpage}
\afterpage{\FloatBarrier}
\usepackage{graphics}


\definecolor{codegreen}{rgb}{0,0.6,0}
\definecolor{codegray}{rgb}{0.5,0.5,0.5}
\definecolor{codepurple}{rgb}{0.58,0,0.82}
\definecolor{backcolour}{rgb}{0.95,0.95,0.92}


\lstdefinestyle{mystyle}{
    language=Java,
    backgroundcolor=\color{backcolour},   
    commentstyle=\color{codegreen},
    keywordstyle=\color{magenta},
    numberstyle=\tiny\color{codegray},
    stringstyle=\color{codepurple},
    basicstyle=\ttfamily\footnotesize ,
    breakatwhitespace=false,         
    breaklines=true,                 
    captionpos=b,                    
    keepspaces=true,                 
    numbers=left,                    
    numbersep=5pt,                  
    showspaces=false,                
    showstringspaces=false,
    showtabs=false,                  
    tabsize=2
}


\lstset{style=mystyle}


\makeglossaries


\title{{\small {A CONCISE MANUAL ON}} \\JAVA PROGRAMMING}  


\author{Kamakshaiah Musunuru \\\tiny{dr.m.kamakshaiah@gmail.com}}
\date{}






\makeatletter

\makeatletter
\count@=\the\catcode`\_ \catcode`\_=8 
\newenvironment{tex2html_wrap}{}{}%
\catcode`\<=12\catcode`\_=\count@
\newcommand{\providedcommand}[1]{\expandafter\providecommand\csname #1\endcsname}%
\newcommand{\renewedcommand}[1]{\expandafter\providecommand\csname #1\endcsname{}%
  \expandafter\renewcommand\csname #1\endcsname}%
\newcommand{\newedenvironment}[1]{\newenvironment{#1}{}{}\renewenvironment{#1}}%
\let\newedcommand\renewedcommand
\let\renewedenvironment\newedenvironment
\makeatother
\let\mathon=$
\let\mathoff=$
\ifx\AtBeginDocument\undefined \newcommand{\AtBeginDocument}[1]{}\fi
\newbox\sizebox
\setlength{\hoffset}{0pt}\setlength{\voffset}{0pt}
\addtolength{\textheight}{\footskip}\setlength{\footskip}{0pt}
\addtolength{\textheight}{\topmargin}\setlength{\topmargin}{0pt}
\addtolength{\textheight}{\headheight}\setlength{\headheight}{0pt}
\addtolength{\textheight}{\headsep}\setlength{\headsep}{0pt}
\setlength{\textwidth}{349pt}
\newwrite\lthtmlwrite
\makeatletter
\let\realnormalsize=\normalsize
\global\topskip=2sp
\def\preveqno{}\let\real@float=\@float \let\realend@float=\end@float
\def\@float{\let\@savefreelist\@freelist\real@float}
\def\liih@math{\ifmmode$\else\bad@math\fi}
\def\end@float{\realend@float\global\let\@freelist\@savefreelist}
\let\real@dbflt=\@dbflt \let\end@dblfloat=\end@float
\let\@largefloatcheck=\relax
\let\if@boxedmulticols=\iftrue
\def\@dbflt{\let\@savefreelist\@freelist\real@dbflt}
\def\adjustnormalsize{\def\normalsize{\mathsurround=0pt \realnormalsize
 \parindent=0pt\abovedisplayskip=0pt\belowdisplayskip=0pt}%
 \def\phantompar{\csname par\endcsname}\normalsize}%
\def\lthtmltypeout#1{{\let\protect\string \immediate\write\lthtmlwrite{#1}}}%
\usepackage[tightpage,active]{preview}
\newbox\lthtmlPageBox
\newdimen\lthtmlCropMarkHeight
\newdimen\lthtmlCropMarkDepth
\long\def\lthtmlTightVBoxA#1{\def\lthtmllabel{#1}
    \setbox\lthtmlPageBox\vbox\bgroup\catcode`\_=8 }%
\long\def\lthtmlTightVBoxZ{\egroup
    \lthtmlCropMarkHeight=\ht\lthtmlPageBox \advance \lthtmlCropMarkHeight 6pt
    \lthtmlCropMarkDepth=\dp\lthtmlPageBox
    \lthtmltypeout{^^J:\lthtmllabel:lthtmlCropMarkHeight:=\the\lthtmlCropMarkHeight}%
    \lthtmltypeout{^^J:\lthtmllabel:lthtmlCropMarkDepth:=\the\lthtmlCropMarkDepth:1ex:=\the \dimexpr 1ex}%
    \begin{preview}\copy\lthtmlPageBox\end{preview}}%
\long\def\lthtmlTightFBoxA#1{\def\lthtmllabel{#1}%
    \adjustnormalsize\setbox\lthtmlPageBox=\vbox\bgroup %
    \let\ifinner=\iffalse \let\)\liih@math %
    \bgroup\catcode`\_=8 }%
\long\def\lthtmlTightFBoxZ{\egroup
    \@next\next\@currlist{}{\def\next{\voidb@x}}%
    \expandafter\box\next\egroup %
    \lthtmlCropMarkHeight=\ht\lthtmlPageBox \advance \lthtmlCropMarkHeight 6pt
    \lthtmlCropMarkDepth=\dp\lthtmlPageBox
    \lthtmltypeout{^^J:\lthtmllabel:lthtmlCropMarkHeight:=\the\lthtmlCropMarkHeight}%
    \lthtmltypeout{^^J:\lthtmllabel:lthtmlCropMarkDepth:=\the\lthtmlCropMarkDepth:1ex:=\the \dimexpr 1ex}%
    \begin{preview}\copy\lthtmlPageBox\end{preview}}%
    \long\def\lthtmlinlinemathA#1#2\lthtmlindisplaymathZ{\lthtmlTightVBoxA{#1}{\hbox\bgroup#2\egroup}\lthtmlTightVBoxZ}
    \def\lthtmlinlineA#1#2\lthtmlinlineZ{\lthtmlTightVBoxA{#1}{\hbox\bgroup#2\egroup}\lthtmlTightVBoxZ}
    \long\def\lthtmldisplayA#1#2\lthtmldisplayZ{\lthtmlTightVBoxA{#1}{#2}\lthtmlTightVBoxZ}
    \long\def\lthtmldisplayB#1#2\lthtmldisplayZ{\\edef\preveqno{(\theequation)}%
        \lthtmlTightVBoxA{#1}{\let\@eqnnum\relax#2}\lthtmlTightVBoxZ}
    \long\def\lthtmlfigureA#1{\let\@savefreelist\@freelist
        \lthtmlTightFBoxA{#1}}
    \long\def\lthtmlfigureZ{
        \lthtmlTightFBoxZ\global\let\@freelist\@savefreelist}
    \long\def\lthtmlpictureA#1{\let\@savefreelist\@freelist
        \lthtmlTightVBoxA{#1}}
    \long\def\lthtmlpictureZ{
        \lthtmlTightVBoxZ\global\let\@freelist\@savefreelist}
\def\lthtmlcheckvsize{\ifdim\ht\sizebox<\vsize 
  \ifdim\wd\sizebox<\hsize\expandafter\hfill\fi \expandafter\vfill
  \else\expandafter\vss\fi}%
\providecommand{\selectlanguage}[1]{}%
\makeatletter \tracingstats = 1 


\begin{document}
\pagestyle{empty}\thispagestyle{empty}\lthtmltypeout{}%
\lthtmltypeout{latex2htmlLength hsize=\the\hsize}\lthtmltypeout{}%
\lthtmltypeout{latex2htmlLength vsize=\the\vsize}\lthtmltypeout{}%
\lthtmltypeout{latex2htmlLength hoffset=\the\hoffset}\lthtmltypeout{}%
\lthtmltypeout{latex2htmlLength voffset=\the\voffset}\lthtmltypeout{}%
\lthtmltypeout{latex2htmlLength topmargin=\the\topmargin}\lthtmltypeout{}%
\lthtmltypeout{latex2htmlLength topskip=\the\topskip}\lthtmltypeout{}%
\lthtmltypeout{latex2htmlLength headheight=\the\headheight}\lthtmltypeout{}%
\lthtmltypeout{latex2htmlLength headsep=\the\headsep}\lthtmltypeout{}%
\lthtmltypeout{latex2htmlLength parskip=\the\parskip}\lthtmltypeout{}%
\lthtmltypeout{latex2htmlLength oddsidemargin=\the\oddsidemargin}\lthtmltypeout{}%
\makeatletter
\if@twoside\lthtmltypeout{latex2htmlLength evensidemargin=\the\evensidemargin}%
\else\lthtmltypeout{latex2htmlLength evensidemargin=\the\oddsidemargin}\fi%
\lthtmltypeout{}%
\makeatother
\setcounter{page}{1}
\onecolumn

% !!! IMAGES START HERE !!!

\stepcounter{chapter}
\stepcounter{section}
\stepcounter{section}
\stepcounter{section}
\stepcounter{section}
\stepcounter{section}
\stepcounter{subsection}
\stepcounter{subsection}
\stepcounter{subsection}
\stepcounter{subsection}
\stepcounter{section}
{\newpage\clearpage
\lthtmlinlinemathA{tex2html_wrap_inline2736}%
$\lstinline{printf}$%
\lthtmlindisplaymathZ
\lthtmlcheckvsize\clearpage}

{\newpage\clearpage
\lthtmlinlinemathA{tex2html_wrap_inline2738}%
$\lstinline{(//)}$%
\lthtmlindisplaymathZ
\lthtmlcheckvsize\clearpage}

{\newpage\clearpage
\lthtmlinlinemathA{tex2html_wrap_inline2740}%
$\lstinline{/*}$%
\lthtmlindisplaymathZ
\lthtmlcheckvsize\clearpage}

{\newpage\clearpage
\lthtmlinlinemathA{tex2html_wrap_inline2742}%
$\lstinline{*/}$%
\lthtmlindisplaymathZ
\lthtmlcheckvsize\clearpage}

{\newpage\clearpage
\lthtmlinlinemathA{tex2html_wrap_inline2744}%
$\lstinline{/**}$%
\lthtmlindisplaymathZ
\lthtmlcheckvsize\clearpage}

\stepcounter{subsection}
{\newpage\clearpage
\lthtmlfigureA{lstlisting85}%
\begin{lstlisting}
public class HelloWorldApp {
    public static void main(String[] args) {
        System.out.println("Hello World!"); // Prints the string to the console.
    }
}
\end{lstlisting}%
\lthtmlfigureZ
\lthtmlcheckvsize\clearpage}

{\newpage\clearpage
\lthtmlinlinemathA{tex2html_wrap_inline2748}%
$\lstinline{.java}$%
\lthtmlindisplaymathZ
\lthtmlcheckvsize\clearpage}

{\newpage\clearpage
\lthtmlinlinemathA{tex2html_wrap_inline2750}%
$\lstinline{HelloWorldApp.java}$%
\lthtmlindisplaymathZ
\lthtmlcheckvsize\clearpage}

{\newpage\clearpage
\lthtmlinlinemathA{tex2html_wrap_inline2752}%
$\lstinline{.class}$%
\lthtmlindisplaymathZ
\lthtmlcheckvsize\clearpage}

{\newpage\clearpage
\lthtmlinlinemathA{tex2html_wrap_inline2754}%
$\lstinline{HelloWorldApp.class}$%
\lthtmlindisplaymathZ
\lthtmlcheckvsize\clearpage}

{\newpage\clearpage
\lthtmlinlinemathA{tex2html_wrap_inline2756}%
$\lstinline{public}$%
\lthtmlindisplaymathZ
\lthtmlcheckvsize\clearpage}

{\newpage\clearpage
\lthtmlinlinemathA{tex2html_wrap_inline2766}%
$\lstinline{private}$%
\lthtmlindisplaymathZ
\lthtmlcheckvsize\clearpage}

{\newpage\clearpage
\lthtmlinlinemathA{tex2html_wrap_inline2768}%
$\lstinline{protected}$%
\lthtmlindisplaymathZ
\lthtmlcheckvsize\clearpage}

{\newpage\clearpage
\lthtmlinlinemathA{tex2html_wrap_inline2770}%
$\lstinline{SecurityException}$%
\lthtmlindisplaymathZ
\lthtmlcheckvsize\clearpage}

{\newpage\clearpage
\lthtmlinlinemathA{tex2html_wrap_inline2772}%
$\lstinline{static}$%
\lthtmlindisplaymathZ
\lthtmlcheckvsize\clearpage}

{\newpage\clearpage
\lthtmlinlinemathA{tex2html_wrap_inline2774}%
$\lstinline{void}$%
\lthtmlindisplaymathZ
\lthtmlcheckvsize\clearpage}

{\newpage\clearpage
\lthtmlinlinemathA{tex2html_wrap_inline2776}%
$\lstinline{System.exit()}$%
\lthtmlindisplaymathZ
\lthtmlcheckvsize\clearpage}

{\newpage\clearpage
\lthtmlinlinemathA{tex2html_wrap_inline2778}%
$\lstinline{String}$%
\lthtmlindisplaymathZ
\lthtmlcheckvsize\clearpage}

{\newpage\clearpage
\lthtmlinlinemathA{tex2html_wrap_inline2780}%
$\lstinline{args}$%
\lthtmlindisplaymathZ
\lthtmlcheckvsize\clearpage}

{\newpage\clearpage
\lthtmlinlinemathA{tex2html_wrap_inline2782}%
$\lstinline{public static void main(String... args)}$%
\lthtmlindisplaymathZ
\lthtmlcheckvsize\clearpage}

{\newpage\clearpage
\lthtmlinlinemathA{tex2html_wrap_inline2788}%
$\lstinline{public static void main(String[])}$%
\lthtmlindisplaymathZ
\lthtmlcheckvsize\clearpage}

{\newpage\clearpage
\lthtmlinlinemathA{tex2html_wrap_inline2790}%
$\lstinline{String[] args}$%
\lthtmlindisplaymathZ
\lthtmlcheckvsize\clearpage}

{\newpage\clearpage
\lthtmlinlinemathA{tex2html_wrap_inline2794}%
$\lstinline{main}$%
\lthtmlindisplaymathZ
\lthtmlcheckvsize\clearpage}

{\newpage\clearpage
\lthtmlinlinemathA{tex2html_wrap_inline2796}%
$\lstinline{System}$%
\lthtmlindisplaymathZ
\lthtmlcheckvsize\clearpage}

{\newpage\clearpage
\lthtmlinlinemathA{tex2html_wrap_inline2798}%
$\lstinline{out}$%
\lthtmlindisplaymathZ
\lthtmlcheckvsize\clearpage}

{\newpage\clearpage
\lthtmlinlinemathA{tex2html_wrap_inline2800}%
$\lstinline{PrintStream}$%
\lthtmlindisplaymathZ
\lthtmlcheckvsize\clearpage}

{\newpage\clearpage
\lthtmlinlinemathA{tex2html_wrap_inline2802}%
$\lstinline{println(String)}$%
\lthtmlindisplaymathZ
\lthtmlcheckvsize\clearpage}

\stepcounter{subsection}
{\newpage\clearpage
\lthtmlfigureA{lstlisting118}%
\begin{lstlisting}
// This is an example of a single line comment using two slashes
\par
/*
 * This is an example of a multiple line comment using the slash and asterisk.
 * This type of comment can be used to hold a lot of information or deactivate
 * code, but it is very important to remember to close the comment.
 */
\par
package fibsandlies;
\par
import java.util.Map;
import java.util.HashMap;
\par
/**
 * This is an example of a Javadoc comment; Javadoc can compile documentation
 * from this text. Javadoc comments must immediately precede the class, method,
 * or field being documented.
 * @author Wikipedia Volunteers
 */
public class FibCalculator extends Fibonacci implements Calculator {
    private static Map<Integer, Integer> memoized = new HashMap<>();
\par
/*
     * The main method written as follows is used by the JVM as a starting point
     * for the program.
     */
    public static void main(String[] args) {
        memoized.put(1, 1);
        memoized.put(2, 1);
        System.out.println(fibonacci(12)); // Get the 12th Fibonacci number and print to console
    }
\par
/**
     * An example of a method written in Java, wrapped in a class.
     * Given a non-negative number FIBINDEX, returns
     * the Nth Fibonacci number, where N equals FIBINDEX.
     * 
     * @param fibIndex The index of the Fibonacci number
     * @return the Fibonacci number
     */
    public static int fibonacci(int fibIndex) {
        if (memoized.containsKey(fibIndex)) return memoized.get(fibIndex);
        else {
            int answer = fibonacci(fibIndex - 1) + fibonacci(fibIndex - 2);
            memoized.put(fibIndex, answer);
            return answer;
        }
    }
}
\par
\end{lstlisting}%
\lthtmlfigureZ
\lthtmlcheckvsize\clearpage}

\stepcounter{section}
\stepcounter{subsection}
\stepcounter{subsubsection}
\stepcounter{subsection}
\stepcounter{subsubsection}
{\newpage\clearpage
\lthtmlinlinemathA{tex2html_wrap_inline2804}%
$\lstinline{service()}$%
\lthtmlindisplaymathZ
\lthtmlcheckvsize\clearpage}

{\newpage\clearpage
\lthtmlinlinemathA{tex2html_wrap_inline2806}%
$\lstinline{HttpServlet}$%
\lthtmlindisplaymathZ
\lthtmlcheckvsize\clearpage}

{\newpage\clearpage
\lthtmlinlinemathA{tex2html_wrap_inline2808}%
$\lstinline{GenericServlet}$%
\lthtmlindisplaymathZ
\lthtmlcheckvsize\clearpage}

{\newpage\clearpage
\lthtmlinlinemathA{tex2html_wrap_inline2814}%
$\lstinline{doGet(), doPost(), doPut(), doDelete()}$%
\lthtmlindisplaymathZ
\lthtmlcheckvsize\clearpage}

{\newpage\clearpage
\lthtmlfigureA{lstlisting140}%
\begin{lstlisting}
import java.io.IOException;
\par
import javax.servlet.ServletConfig;
import javax.servlet.ServletException;
import javax.servlet.http.HttpServlet;
import javax.servlet.http.HttpServletRequest;
import javax.servlet.http.HttpServletResponse;
\par
public class ServletLifeCycleExample extends HttpServlet {
\par
private Integer sharedCounter;                 
\par
@Override
    public void init(final ServletConfig config) throws ServletException {
        super.init(config);
        getServletContext().log("init() called");
        sharedCounter = 0;
    }
\par
@Override
    protected void service(final HttpServletRequest request, final HttpServletResponse response) throws ServletException, IOException {
        getServletContext().log("service() called");
\par
int localCounter;                       
\par
synchronized (sharedCounter) {
              sharedCounter++;                  
\par
localCounter = sharedCounter;       
        }
\par
response.getWriter().write("Incrementing the count to " + localCounter);  // accessing a local variable
    }
\par
@Override
    public void destroy() {
        getServletContext().log("destroy() called");
    }
}
\end{lstlisting}%
\lthtmlfigureZ
\lthtmlcheckvsize\clearpage}

\stepcounter{subsection}
{\newpage\clearpage
\lthtmlfigureA{lstlisting146}%
\begin{lstlisting}
<p>Counting to three:</p>
<% for (int i=1; i<4; i++) { %>
    <p>This number is <%= i %>.</p>
<% } %>
<p>OK.</p>
\end{lstlisting}%
\lthtmlfigureZ
\lthtmlcheckvsize\clearpage}

{\newpage\clearpage
\lthtmlfigureA{lstlisting148}%
\begin{lstlisting}
Counting to three:
\par
This number is 1.
\par
This number is 2.
\par
This number is 3.
\par
OK.
\end{lstlisting}%
\lthtmlfigureZ
\lthtmlcheckvsize\clearpage}

\stepcounter{subsection}
\stepcounter{subsubsection}
{\newpage\clearpage
\lthtmlfigureA{lstlisting153}%
\begin{lstlisting}
// Hello.java (Java SE 5)
import javax.swing.*;
\par
public class Hello extends JFrame {
    public Hello() {
        super("hello");
        this.setDefaultCloseOperation(WindowConstants.EXIT_ON_CLOSE);
        this.add(new JLabel("Hello, world!"));
        this.pack();
        this.setVisible(true);
    }
\par
public static void main(final String[] args) {
        new Hello();
    }
}
\end{lstlisting}%
\lthtmlfigureZ
\lthtmlcheckvsize\clearpage}

{\newpage\clearpage
\lthtmlinlinemathA{tex2html_wrap_inline2818}%
$\lstinline{import}$%
\lthtmlindisplaymathZ
\lthtmlcheckvsize\clearpage}

{\newpage\clearpage
\lthtmlinlinemathA{tex2html_wrap_inline2820}%
$\lstinline{javax.swing}$%
\lthtmlindisplaymathZ
\lthtmlcheckvsize\clearpage}

{\newpage\clearpage
\lthtmlinlinemathA{tex2html_wrap_inline2822}%
$\lstinline{Hello}$%
\lthtmlindisplaymathZ
\lthtmlcheckvsize\clearpage}

{\newpage\clearpage
\lthtmlinlinemathA{tex2html_wrap_inline2824}%
$\lstinline{extends}$%
\lthtmlindisplaymathZ
\lthtmlcheckvsize\clearpage}

{\newpage\clearpage
\lthtmlinlinemathA{tex2html_wrap_inline2826}%
$\lstinline{JFrame}$%
\lthtmlindisplaymathZ
\lthtmlcheckvsize\clearpage}

{\newpage\clearpage
\lthtmlinlinemathA{tex2html_wrap_inline2830}%
$\lstinline{Hello()}$%
\lthtmlindisplaymathZ
\lthtmlcheckvsize\clearpage}

{\newpage\clearpage
\lthtmlinlinemathA{tex2html_wrap_inline2832}%
$\lstinline{setDefaultCloseOperation(int)}$%
\lthtmlindisplaymathZ
\lthtmlcheckvsize\clearpage}

{\newpage\clearpage
\lthtmlinlinemathA{tex2html_wrap_inline2836}%
$\lstinline{WindowConstants.EXIT_ON_CLOSE}$%
\lthtmlindisplaymathZ
\lthtmlcheckvsize\clearpage}

{\newpage\clearpage
\lthtmlinlinemathA{tex2html_wrap_inline2840}%
$\lstinline{JLabel}$%
\lthtmlindisplaymathZ
\lthtmlcheckvsize\clearpage}

{\newpage\clearpage
\lthtmlinlinemathA{tex2html_wrap_inline2842}%
$\lstinline{add(Component)}$%
\lthtmlindisplaymathZ
\lthtmlcheckvsize\clearpage}

{\newpage\clearpage
\lthtmlinlinemathA{tex2html_wrap_inline2844}%
$\lstinline{Container}$%
\lthtmlindisplaymathZ
\lthtmlcheckvsize\clearpage}

{\newpage\clearpage
\lthtmlinlinemathA{tex2html_wrap_inline2846}%
$\lstinline{pack()}$%
\lthtmlindisplaymathZ
\lthtmlcheckvsize\clearpage}

{\newpage\clearpage
\lthtmlinlinemathA{tex2html_wrap_inline2848}%
$\lstinline{Window}$%
\lthtmlindisplaymathZ
\lthtmlcheckvsize\clearpage}

{\newpage\clearpage
\lthtmlinlinemathA{tex2html_wrap_inline2850}%
$\lstinline{main()}$%
\lthtmlindisplaymathZ
\lthtmlcheckvsize\clearpage}

{\newpage\clearpage
\lthtmlinlinemathA{tex2html_wrap_inline2852}%
$\lstinline{setVisible(boolean)}$%
\lthtmlindisplaymathZ
\lthtmlcheckvsize\clearpage}

{\newpage\clearpage
\lthtmlinlinemathA{tex2html_wrap_inline2854}%
$\lstinline{Component}$%
\lthtmlindisplaymathZ
\lthtmlcheckvsize\clearpage}

{\newpage\clearpage
\lthtmlinlinemathA{tex2html_wrap_inline2856}%
$\lstinline{true}$%
\lthtmlindisplaymathZ
\lthtmlcheckvsize\clearpage}

{\newpage\clearpage
\lthtmlfigureA{lstlisting184}%
\begin{lstlisting}
import java.awt.FlowLayout;
import javax.swing.JButton;
import javax.swing.JFrame;
import javax.swing.JLabel;
import javax.swing.WindowConstants;
import javax.swing.SwingUtilities;
\par
public class SwingExample implements Runnable {
\par
@Override
    public void run() {
        // Create the window
        JFrame f = new JFrame("Hello, !");
        // Sets the behavior for when the window is closed
        f.setDefaultCloseOperation(WindowConstants.EXIT_ON_CLOSE);
        // Add a layout manager so that the button is not placed on top of the label
        f.setLayout(new FlowLayout());
        // Add a label and a button
        f.add(new JLabel("Hello, world!"));
        f.add(new JButton("Press me!"));
        // Arrange the components inside the window
        f.pack();
        // By default, the window is not visible. Make it visible.
        f.setVisible(true);
    }
\par
public static void main(String[] args) {
        SwingExample se = new SwingExample();
        // Schedules the application to be run at the correct time in the event queue.
        SwingUtilities.invokeLater(se);
    }
\par
}
\end{lstlisting}%
\lthtmlfigureZ
\lthtmlcheckvsize\clearpage}

{\newpage\clearpage
\lthtmlinlinemathA{tex2html_wrap_inline2858}%
$\lstinline{SwingUtilities.invokeLater(Runnable))}$%
\lthtmlindisplaymathZ
\lthtmlcheckvsize\clearpage}

\stepcounter{subsection}
\stepcounter{subsubsection}
\stepcounter{subsubsection}
{\newpage\clearpage
\lthtmlfigureA{lstlisting196}%
\begin{lstlisting}
package javafxtuts;
\par
import javafx.application.Application;
import javafx.event.ActionEvent;
import javafx.event.EventHandler;
import javafx.scene.Scene;
import javafx.scene.control.Button;
import javafx.scene.layout.StackPane;
import javafx.stage.Stage;
\par
public class JavaFxTuts extends Application {
    public JavaFxTuts() {
        //Optional constructor
    }
    @Override
    public void init() {
         //By default this does nothing, but it
         //can carry out code to set up your app.
         //It runs once before the start method,
         //and after the constructor.
    }
\par
@Override
    public void start(Stage primaryStage) {
        // Creating the Java button
        final Button button = new Button();
        // Setting text to button
        button.setText("Hello World");
        // Registering a handler for button
        button.setOnAction((ActionEvent event) -> {
            // Printing Hello World! to the console
            System.out.println("Hello World!");
        });
        // Initializing the StackPane class
        final StackPane root = new StackPane();
        // Adding all the nodes to the StackPane
        root.getChildren().add(button);
        // Creating a scene object
        final Scene scene = new Scene(root, 300, 250);
        // Adding the title to the window (primaryStage)
        primaryStage.setTitle("Hello World!");
        primaryStage.setScene(scene);
        // Show the window(primaryStage)
        primaryStage.show();
    }
    @Override
    public void stop() {
        //By default this does nothing
        //It runs if the user clicks the go-away button
        //closing the window or if Platform.exit() is called.
        //Use Platform.exit() instead of System.exit(0).
        //This is where you should offer to save any unsaved
        //stuff that the user may have generated.
    }
\par
/**
     * Main function that opens the "Hello World!" window
     * 
     * @param arguments the command line arguments
     */
    public static void main(final String[] arguments) {
        launch(arguments);
    }
}
\end{lstlisting}%
\lthtmlfigureZ
\lthtmlcheckvsize\clearpage}

\stepcounter{section}
\stepcounter{chapter}
\stepcounter{section}
{\newpage\clearpage
\lthtmlfigureA{lstlisting226}%
\begin{lstlisting}
// Your First Program
\par
class HelloWorld {
    public static void main(String[] args) {
        System.out.println("Hello, World!"); 
    }
}
\end{lstlisting}%
\lthtmlfigureZ
\lthtmlcheckvsize\clearpage}

{\newpage\clearpage
\lthtmlfigureA{lstlisting232}%
\begin{lstlisting}
javac HelloWorld.java
java HelloWorld
\end{lstlisting}%
\lthtmlfigureZ
\lthtmlcheckvsize\clearpage}

{\newpage\clearpage
\lthtmlinlinemathA{tex2html_wrap_inline2860}%
$\lstinline{java HelloWorld}$%
\lthtmlindisplaymathZ
\lthtmlcheckvsize\clearpage}

\stepcounter{subsection}
{\newpage\clearpage
\lthtmlfigureA{lstlisting244}%
\begin{lstlisting}
class HelloWorld {
... .. ...
}
\end{lstlisting}%
\lthtmlfigureZ
\lthtmlcheckvsize\clearpage}

{\newpage\clearpage
\lthtmlfigureA{lstlisting250}%
\begin{lstlisting}
public static void main(String[] args) {
... .. ...
}
\end{lstlisting}%
\lthtmlfigureZ
\lthtmlcheckvsize\clearpage}

\stepcounter{section}
\stepcounter{section}
\stepcounter{subsection}
\stepcounter{subsubsection}
{\newpage\clearpage
\lthtmlfigureA{lstlisting318}%
\begin{lstlisting}
E:\work\java\eclipse\HelloWorld\src>dir
 Volume in drive E has no label.
 Volume Serial Number is ####-####
\par
Directory of E:\work\java\eclipse\HelloWorld\src
.
..
179 helloworld.java
		p1
		p2
\par
1 File(s)            179 bytes
    4 Dir(s)  85,037,547,520 bytes free
\end{lstlisting}%
\lthtmlfigureZ
\lthtmlcheckvsize\clearpage}

{\newpage\clearpage
\lthtmlfigureA{lstlisting331}%
\begin{lstlisting}
// package p1 
package p1;
\par
public class A {
	public void display(){
		System.out.println("Class A is called here!");
	}	
\end{lstlisting}%
\lthtmlfigureZ
\lthtmlcheckvsize\clearpage}

{\newpage\clearpage
\lthtmlfigureA{lstlisting334}%
\begin{lstlisting}
// package p2 
package p2;
import p1.*;
\par
public class B {
\par
static void display(){
		System.out.println("This is class B!");
	}
	public static void main(String[] args){
		A obj = new A();
		obj.display();
		display();
	}
}
\end{lstlisting}%
\lthtmlfigureZ
\lthtmlcheckvsize\clearpage}

\stepcounter{subsubsection}
{\newpage\clearpage
\lthtmlfigureA{lstlisting348}%
\begin{lstlisting}
//Java program to illustrate error while  
//using class from different package with 
//private modifier 
package p1; 
\par
class A 
{ 
   private void display() 
    { 
        System.out.println("This is class A of p1"); 
    } 
} 
\par
class B 
{ 
   public static void main(String args[]) 
      { 
          A obj = new A(); 
          //trying to access private method of another class 
          obj.display(); 
      } 
} 
\end{lstlisting}%
\lthtmlfigureZ
\lthtmlcheckvsize\clearpage}

\stepcounter{subsubsection}
{\newpage\clearpage
\lthtmlfigureA{lstlisting368}%
\begin{lstlisting}
//Java program to illustrate 
//protected modifier 
package p1; 
\par
//Class A 
public class A 
{ 
   protected void display() 
    { 
        System.out.println("This is class A of p1"); 
    } 
}
\end{lstlisting}%
\lthtmlfigureZ
\lthtmlcheckvsize\clearpage}

{\newpage\clearpage
\lthtmlfigureA{lstlisting371}%
\begin{lstlisting}
//Java program to illustrate 
//protected modifier 
package p2; 
import p1.*; //importing all classes in package p1 
\par
//Class B is subclass of A 
class B extends A 
// class inheritance
{ 
   public static void main(String args[]) 
   {   
       B obj = new B();   
       obj.display();   
   }   
\par
} 
\end{lstlisting}%
\lthtmlfigureZ
\lthtmlcheckvsize\clearpage}

\stepcounter{subsection}
\stepcounter{subsubsection}
\stepcounter{subsubsection}
{\newpage\clearpage
\lthtmlfigureA{lstlisting383}%
\begin{lstlisting}
package teststatic;
\par
public class TestStatic {
\par
static void callStatic(){
		System.out.println("Static method called in main method!");
	}
\par
void callNonStatic(){
		System.out.println("Non-static method called in main method! using 'instance'");
	}
\par
public static void main(String[] args) {
\par
// this static method
		callStatic();
\par
// this is non-staic method
		TestStatic obj = new TestStatic();
		obj.callNonStatic();
		// this is also possible
		obj.callStatic();
	}
}
\end{lstlisting}%
\lthtmlfigureZ
\lthtmlcheckvsize\clearpage}

\stepcounter{subsubsection}
\stepcounter{subsubsection}
{\newpage\clearpage
\lthtmlfigureA{lstlisting393}%
\begin{lstlisting}
package teststatic;
\par
class Parent { 
    void show() 
    { 
        System.out.println("Parent"); 
    } 
} 
\par
// Parent inherit in Child class 
class Child extends Parent { 
\par
// override show() of Parent 
    void show() 
    { 
        System.out.println("Child"); 
    } 
} 
\par
class TestStatic { 
    public static void main(String[] args) 
    { 
        Parent p = new Parent(); 
        // calling Parent's show() 
        p.show(); 
\par
Parent c = new Child(); 
        // calling Child's show() 
        c.show(); 
    } 
} 
\end{lstlisting}%
\lthtmlfigureZ
\lthtmlcheckvsize\clearpage}

{\newpage\clearpage
\lthtmlfigureA{lstlisting398}%
\begin{lstlisting}
package teststatic;
\par
class Parent { 
    static void show() 
    { 
        System.out.println("Parent"); 
    } 
} 
\par
// Parent inherit in Child class 
class Child extends Parent { 
\par
// override show() of Parent 
    void show() 
    { 
        System.out.println("Child"); 
    } 
} 
\par
class TestStatic { 
    public static void main(String[] args) 
    { 
        Parent p = new Parent(); 
        // calling Parent's show() 
        p.show(); 
\par
Parent c = new Child(); 
        // calling Child's show() 
        c.show(); 
    } 
} 
\end{lstlisting}%
\lthtmlfigureZ
\lthtmlcheckvsize\clearpage}

{\newpage\clearpage
\lthtmlfigureA{lstlisting403}%
\begin{lstlisting}
Multiple markers at this line
	- This instance method cannot override the static method 
	 from Parent
\end{lstlisting}%
\lthtmlfigureZ
\lthtmlcheckvsize\clearpage}

\stepcounter{subsection}
{\newpage\clearpage
\lthtmlfigureA{lstlisting406}%
\begin{lstlisting}
int score;
\end{lstlisting}%
\lthtmlfigureZ
\lthtmlcheckvsize\clearpage}

{\newpage\clearpage
\lthtmlinlinemathA{tex2html_wrap_inline2862}%
$\lstinline{int}$%
\lthtmlindisplaymathZ
\lthtmlcheckvsize\clearpage}

{\newpage\clearpage
\lthtmlinlinemathA{tex2html_wrap_inline2864}%
$\lstinline{score}$%
\lthtmlindisplaymathZ
\lthtmlcheckvsize\clearpage}

{\newpage\clearpage
\lthtmlinlinemathA{tex2html_wrap_inline2866}%
$\lstinline{int, for, class}$%
\lthtmlindisplaymathZ
\lthtmlcheckvsize\clearpage}

\stepcounter{subsubsection}
{\newpage\clearpage
\lthtmlinlinemathA{tex2html_wrap_inline2868}%
$\lstinline{true, false}$%
\lthtmlindisplaymathZ
\lthtmlcheckvsize\clearpage}

{\newpage\clearpage
\lthtmlinlinemathA{tex2html_wrap_inline2870}%
$\lstinline{null}$%
\lthtmlindisplaymathZ
\lthtmlcheckvsize\clearpage}

\stepcounter{subsection}
{\newpage\clearpage
\lthtmlfigureA{lstlisting424}%
\begin{lstlisting}
int float;
\end{lstlisting}%
\lthtmlfigureZ
\lthtmlcheckvsize\clearpage}

{\newpage\clearpage
\lthtmlinlinemathA{tex2html_wrap_inline2874}%
$\lstinline{float}$%
\lthtmlindisplaymathZ
\lthtmlcheckvsize\clearpage}

\stepcounter{subsection}
\stepcounter{section}
{\newpage\clearpage
\lthtmlfigureA{lstlisting438}%
\begin{lstlisting}
int speedLimit = 80;
\end{lstlisting}%
\lthtmlfigureZ
\lthtmlcheckvsize\clearpage}

{\newpage\clearpage
\lthtmlinlinemathA{tex2html_wrap_inline2876}%
$\lstinline{speedLimit}$%
\lthtmlindisplaymathZ
\lthtmlcheckvsize\clearpage}

{\newpage\clearpage
\lthtmlfigureA{lstlisting443}%
\begin{lstlisting}
int speedLimit;
speedLimit = 80;
\end{lstlisting}%
\lthtmlfigureZ
\lthtmlcheckvsize\clearpage}

{\newpage\clearpage
\lthtmlfigureA{lstlisting446}%
\begin{lstlisting}
int speedLimit = 80;
... ... ...
speedLimit = 90; 
\end{lstlisting}%
\lthtmlfigureZ
\lthtmlcheckvsize\clearpage}

{\newpage\clearpage
\lthtmlfigureA{lstlisting448}%
\begin{lstlisting}
int speedLimit = 80;
... ... ...
float speedLimit;
\end{lstlisting}%
\lthtmlfigureZ
\lthtmlcheckvsize\clearpage}

\stepcounter{subsection}
\stepcounter{subsection}
{\newpage\clearpage
\lthtmlfigureA{lstlisting464}%
\begin{lstlisting}
int speed;
\end{lstlisting}%
\lthtmlfigureZ
\lthtmlcheckvsize\clearpage}

{\newpage\clearpage
\lthtmlinlinemathA{tex2html_wrap_inline2882}%
$\lstinline{speed}$%
\lthtmlindisplaymathZ
\lthtmlcheckvsize\clearpage}

\stepcounter{subsection}
\stepcounter{subsubsection}
{\newpage\clearpage
\lthtmlinlinemathA{tex2html_wrap_inline2888}%
$\lstinline{boolean}$%
\lthtmlindisplaymathZ
\lthtmlcheckvsize\clearpage}

{\newpage\clearpage
\lthtmlinlinemathA{tex2html_wrap_inline2892}%
$\lstinline{false}$%
\lthtmlindisplaymathZ
\lthtmlcheckvsize\clearpage}

{\newpage\clearpage
\lthtmlfigureA{lstlisting475}%
\begin{lstlisting}
class BooleanExample {
    public static void main(String[] args) {
\par
boolean flag = true;
        System.out.println(flag);
    }
}
\end{lstlisting}%
\lthtmlfigureZ
\lthtmlcheckvsize\clearpage}

\stepcounter{subsubsection}
{\newpage\clearpage
\lthtmlinlinemathA{tex2html_wrap_inline2896}%
$\lstinline{byte}$%
\lthtmlindisplaymathZ
\lthtmlcheckvsize\clearpage}

{\newpage\clearpage
\lthtmlinlinemathA{tex2html_wrap_inline2898}%
$-128$%
\lthtmlindisplaymathZ
\lthtmlcheckvsize\clearpage}

{\newpage\clearpage
\lthtmlinlinemathA{tex2html_wrap_inline2900}%
$127$%
\lthtmlindisplaymathZ
\lthtmlcheckvsize\clearpage}

{\newpage\clearpage
\lthtmlinlinemathA{tex2html_wrap_inline2904}%
$[-128, 127]$%
\lthtmlindisplaymathZ
\lthtmlcheckvsize\clearpage}

{\newpage\clearpage
\lthtmlfigureA{lstlisting481}%
\begin{lstlisting}
class ByteExample {
    public static void main(String[] args) {
\par
byte range;
        range = 124;
        System.out.println(range);
    }
}
\end{lstlisting}%
\lthtmlfigureZ
\lthtmlcheckvsize\clearpage}

\stepcounter{subsubsection}
{\newpage\clearpage
\lthtmlinlinemathA{tex2html_wrap_inline2906}%
$\lstinline{short}$%
\lthtmlindisplaymathZ
\lthtmlcheckvsize\clearpage}

{\newpage\clearpage
\lthtmlinlinemathA{tex2html_wrap_inline2908}%
$-32768 to 32767$%
\lthtmlindisplaymathZ
\lthtmlcheckvsize\clearpage}

{\newpage\clearpage
\lthtmlinlinemathA{tex2html_wrap_inline2910}%
$[-32768, 32767]$%
\lthtmlindisplaymathZ
\lthtmlcheckvsize\clearpage}

{\newpage\clearpage
\lthtmlfigureA{lstlisting486}%
\begin{lstlisting}
class ShortExample {
    public static void main(String[] args) {
\par
short temperature;
        temperature = -200;
        System.out.println(temperature);
    }
}
\end{lstlisting}%
\lthtmlfigureZ
\lthtmlcheckvsize\clearpage}

\stepcounter{subsubsection}
{\newpage\clearpage
\lthtmlinlinemathA{tex2html_wrap_inline2914}%
$-2^31$%
\lthtmlindisplaymathZ
\lthtmlcheckvsize\clearpage}

{\newpage\clearpage
\lthtmlinlinemathA{tex2html_wrap_inline2916}%
$2^{31}-1$%
\lthtmlindisplaymathZ
\lthtmlcheckvsize\clearpage}

{\newpage\clearpage
\lthtmlinlinemathA{tex2html_wrap_inline2918}%
$2^{32}-1$%
\lthtmlindisplaymathZ
\lthtmlcheckvsize\clearpage}

{\newpage\clearpage
\lthtmlfigureA{lstlisting493}%
\begin{lstlisting}
class IntExample {
    public static void main(String[] args) {
\par
int range = -4250000;
        System.out.println(range);
    }
}
\end{lstlisting}%
\lthtmlfigureZ
\lthtmlcheckvsize\clearpage}

\stepcounter{subsubsection}
{\newpage\clearpage
\lthtmlinlinemathA{tex2html_wrap_inline2920}%
$\lstinline{long}$%
\lthtmlindisplaymathZ
\lthtmlcheckvsize\clearpage}

{\newpage\clearpage
\lthtmlinlinemathA{tex2html_wrap_inline2922}%
$-2^63$%
\lthtmlindisplaymathZ
\lthtmlcheckvsize\clearpage}

{\newpage\clearpage
\lthtmlinlinemathA{tex2html_wrap_inline2924}%
$2^{63}-1$%
\lthtmlindisplaymathZ
\lthtmlcheckvsize\clearpage}

{\newpage\clearpage
\lthtmlinlinemathA{tex2html_wrap_inline2926}%
$2^{64}-1$%
\lthtmlindisplaymathZ
\lthtmlcheckvsize\clearpage}

{\newpage\clearpage
\lthtmlfigureA{lstlisting500}%
\begin{lstlisting}
class LongExample {
    public static void main(String[] args) {
\par
long range = -42332200000L;
        System.out.println(range);
    }
}
\end{lstlisting}%
\lthtmlfigureZ
\lthtmlcheckvsize\clearpage}

{\newpage\clearpage
\lthtmlinlinemathA{tex2html_wrap_inline2928}%
$-42332200000$%
\lthtmlindisplaymathZ
\lthtmlcheckvsize\clearpage}

\stepcounter{subsubsection}
{\newpage\clearpage
\lthtmlinlinemathA{tex2html_wrap_inline2930}%
$\lstinline{double}$%
\lthtmlindisplaymathZ
\lthtmlcheckvsize\clearpage}

{\newpage\clearpage
\lthtmlfigureA{lstlisting506}%
\begin{lstlisting}
class DoubleExample {
    public static void main(String[] args) {
\par
double number = -42.3;
        System.out.println(number);
    }
}
\end{lstlisting}%
\lthtmlfigureZ
\lthtmlcheckvsize\clearpage}

\stepcounter{subsubsection}
{\newpage\clearpage
\lthtmlfigureA{lstlisting511}%
\begin{lstlisting}
class FloatExample {
    public static void main(String[] args) {
\par
float number = -42.3f;
        System.out.println(number);
    }
}
\par
\end{lstlisting}%
\lthtmlfigureZ
\lthtmlcheckvsize\clearpage}

{\newpage\clearpage
\lthtmlinlinemathA{tex2html_wrap_inline2934}%
$-42.3f$%
\lthtmlindisplaymathZ
\lthtmlcheckvsize\clearpage}

{\newpage\clearpage
\lthtmlinlinemathA{tex2html_wrap_inline2936}%
$-42.3$%
\lthtmlindisplaymathZ
\lthtmlcheckvsize\clearpage}

\stepcounter{subsubsection}
{\newpage\clearpage
\lthtmlinlinemathA{tex2html_wrap_inline2946}%
$'\symbol{92}u0000'$%
\lthtmlindisplaymathZ
\lthtmlcheckvsize\clearpage}

{\newpage\clearpage
\lthtmlinlinemathA{tex2html_wrap_inline2948}%
$'\symbol{92}uffff'$%
\lthtmlindisplaymathZ
\lthtmlcheckvsize\clearpage}

{\newpage\clearpage
\lthtmlfigureA{lstlisting522}%
\begin{lstlisting}
class CharExample {
    public static void main(String[] args) {
\par
char letter = '\u0051';
        System.out.println(letter);
    }
}
\end{lstlisting}%
\lthtmlfigureZ
\lthtmlcheckvsize\clearpage}

{\newpage\clearpage
\lthtmlinlinemathA{tex2html_wrap_inline2952}%
$'\symbol{92}u0051'$%
\lthtmlindisplaymathZ
\lthtmlcheckvsize\clearpage}

{\newpage\clearpage
\lthtmlfigureA{lstlisting528}%
\begin{lstlisting}
class CharExample {
    public static void main(String[] args) {
\par
char letter1 = '9';
        System.out.println(letter1);
\par
char letter2 = 65;
        System.out.println(letter2);
\par
}
}
\end{lstlisting}%
\lthtmlfigureZ
\lthtmlcheckvsize\clearpage}

{\newpage\clearpage
\lthtmlinlinemathA{tex2html_wrap_inline2954}%
$\lstinline{letter1}$%
\lthtmlindisplaymathZ
\lthtmlcheckvsize\clearpage}

{\newpage\clearpage
\lthtmlinlinemathA{tex2html_wrap_inline2958}%
$\lstinline{letter2}$%
\lthtmlindisplaymathZ
\lthtmlcheckvsize\clearpage}

\stepcounter{subsubsection}
{\newpage\clearpage
\lthtmlfigureA{lstlisting535}%
\begin{lstlisting}
myString = "Programming is awesome";
\end{lstlisting}%
\lthtmlfigureZ
\lthtmlcheckvsize\clearpage}

\stepcounter{section}
{\newpage\clearpage
\lthtmlfigureA{lstlisting538}%
\begin{lstlisting}
boolean flag = false;
\end{lstlisting}%
\lthtmlfigureZ
\lthtmlcheckvsize\clearpage}

{\newpage\clearpage
\lthtmlinlinemathA{tex2html_wrap_inline2962}%
$\lstinline{flag}$%
\lthtmlindisplaymathZ
\lthtmlcheckvsize\clearpage}

{\newpage\clearpage
\lthtmlinlinemathA{tex2html_wrap_inline2966}%
$1.5, 4, true, '\symbol{92}u0050'$%
\lthtmlindisplaymathZ
\lthtmlcheckvsize\clearpage}

{\newpage\clearpage
\lthtmlinlinemathA{tex2html_wrap_inline2970}%
$-5, 'a', true$%
\lthtmlindisplaymathZ
\lthtmlcheckvsize\clearpage}

\stepcounter{subsection}
{\newpage\clearpage
\lthtmlfigureA{lstlisting553}%
\begin{lstlisting}
// Error! literal 42332200000 of type int is out of range
int myVariable1 = 42332200000;// 42332200000L is of type long, and it's not out of range
long myVariable2 = 42332200000L;
\end{lstlisting}%
\lthtmlfigureZ
\lthtmlcheckvsize\clearpage}

{\newpage\clearpage
\lthtmlinlinemathA{tex2html_wrap_inline2972}%
$0x$%
\lthtmlindisplaymathZ
\lthtmlcheckvsize\clearpage}

{\newpage\clearpage
\lthtmlinlinemathA{tex2html_wrap_inline2974}%
$0b$%
\lthtmlindisplaymathZ
\lthtmlcheckvsize\clearpage}

{\newpage\clearpage
\lthtmlfigureA{lstlisting555}%
\begin{lstlisting}
// decimal
int decNumber = 34;
int hexNumber = 0x2F; // 0x represents hexadecimal
int binNumber = 0b10010; // 0b represents binary
\end{lstlisting}%
\lthtmlfigureZ
\lthtmlcheckvsize\clearpage}

\stepcounter{subsection}
{\newpage\clearpage
\lthtmlfigureA{lstlisting566}%
\begin{lstlisting}
class DoubleExample {
    public static void main(String[] args) {
\par
double myDouble = 3.4;
        float myFloat = 3.4F;
\par
// 3.445*10^2
        double myDoubleScientific = 3.445e2;
\par
System.out.println(myDouble);
        System.out.println(myFloat);
        System.out.println(myDoubleScientific);
    }
}
\end{lstlisting}%
\lthtmlfigureZ
\lthtmlcheckvsize\clearpage}

\stepcounter{subsection}
{\newpage\clearpage
\lthtmlinlinemathA{tex2html_wrap_inline2976}%
$\lstinline{char}$%
\lthtmlindisplaymathZ
\lthtmlcheckvsize\clearpage}

{\newpage\clearpage
\lthtmlinlinemathA{tex2html_wrap_inline2978}%
$'a', '\symbol{92}u0111'$%
\lthtmlindisplaymathZ
\lthtmlcheckvsize\clearpage}

{\newpage\clearpage
\lthtmlinlinemathA{tex2html_wrap_inline2982}%
$\symbol{92}b$%
\lthtmlindisplaymathZ
\lthtmlcheckvsize\clearpage}

{\newpage\clearpage
\lthtmlinlinemathA{tex2html_wrap_inline2984}%
$\symbol{92}t$%
\lthtmlindisplaymathZ
\lthtmlcheckvsize\clearpage}

{\newpage\clearpage
\lthtmlinlinemathA{tex2html_wrap_inline2986}%
$\symbol{92}n$%
\lthtmlindisplaymathZ
\lthtmlcheckvsize\clearpage}

{\newpage\clearpage
\lthtmlinlinemathA{tex2html_wrap_inline2988}%
$\symbol{92}f$%
\lthtmlindisplaymathZ
\lthtmlcheckvsize\clearpage}

{\newpage\clearpage
\lthtmlinlinemathA{tex2html_wrap_inline2990}%
$\symbol{92}r$%
\lthtmlindisplaymathZ
\lthtmlcheckvsize\clearpage}

{\newpage\clearpage
\lthtmlinlinemathA{tex2html_wrap_inline2992}%
$\symbol{92}"$%
\lthtmlindisplaymathZ
\lthtmlcheckvsize\clearpage}

{\newpage\clearpage
\lthtmlinlinemathA{tex2html_wrap_inline2994}%
$\symbol{92}'$%
\lthtmlindisplaymathZ
\lthtmlcheckvsize\clearpage}

{\newpage\clearpage
\lthtmlinlinemathA{tex2html_wrap_inline2996}%
$\symbol{92}\symbol{92}$%
\lthtmlindisplaymathZ
\lthtmlcheckvsize\clearpage}

{\newpage\clearpage
\lthtmlfigureA{lstlisting584}%
\begin{lstlisting}
class DoubleExample {
    public static void main(String[] args) {
\par
char myChar = 'g';
        char newLine = '\n';
        String myString = "Java 8";
\par
System.out.println(myChar);
        System.out.println(newLine);
        System.out.println(myString);
    }
}
\end{lstlisting}%
\lthtmlfigureZ
\lthtmlcheckvsize\clearpage}

\stepcounter{section}
\stepcounter{subsection}
{\newpage\clearpage
\lthtmlfigureA{lstlisting590}%
\begin{lstlisting}
int age;
age = 5;
\end{lstlisting}%
\lthtmlfigureZ
\lthtmlcheckvsize\clearpage}

\stepcounter{subsubsection}
{\newpage\clearpage
\lthtmlfigureA{lstlisting595}%
\begin{lstlisting}
class AssignmentOperator {
    public static void main(String[] args) {
\par
int number1, number2;
\par
// Assigning 5 to number1 
        number1 = 5;
        System.out.println(number1);
\par
// Assigning value of variable number2 to number1
        number2 = number1;
        System.out.println(number2);
    }
}
\end{lstlisting}%
\lthtmlfigureZ
\lthtmlcheckvsize\clearpage}

\stepcounter{subsection}
\stepcounter{subsubsection}
{\newpage\clearpage
\lthtmlfigureA{lstlisting607}%
\begin{lstlisting}
class ArithmeticOperator {
    public static void main(String[] args) {
\par
double number1 = 12.5, number2 = 3.5, result;
\par
// Using addition operator
        result = number1 + number2;
        System.out.println("number1 + number2 = " + result);
\par
// Using subtraction operator
        result = number1 - number2;
        System.out.println("number1 - number2 = " + result);
\par
// Using multiplication operator
        result = number1 * number2;
        System.out.println("number1 * number2 = " + result);
\par
// Using division operator
        result = number1 / number2;
        System.out.println("number1 / number2 = " + result);
\par
// Using remainder operator
        result = number1 % number2;
        System.out.println("number1 % number2 = " + result);
    }
}
\end{lstlisting}%
\lthtmlfigureZ
\lthtmlcheckvsize\clearpage}

{\newpage\clearpage
\lthtmlfigureA{lstlisting610}%
\begin{lstlisting}
result = number1 + 5.2;
result = 2.3 + 4.5;
number2 = number1 -2.9;
\end{lstlisting}%
\lthtmlfigureZ
\lthtmlcheckvsize\clearpage}

{\newpage\clearpage
\lthtmlfigureA{lstlisting612}%
\begin{lstlisting}
class ArithmeticOperator {
    public static void main(String[] args) {
\par
String start, middle, end, result;
\par
start = "Talk is cheap. ";
        middle = "Show me the code. ";
        end = "- Linus Torvalds";
\par
result = start + middle + end;
        System.out.println(result);
    }
}
\end{lstlisting}%
\lthtmlfigureZ
\lthtmlcheckvsize\clearpage}

\stepcounter{subsection}
\stepcounter{subsubsection}
{\newpage\clearpage
\lthtmlfigureA{lstlisting626}%
\begin{lstlisting}
class UnaryOperator {
    public static void main(String[] args) {
\par
double number = 5.2, resultNumber;
        boolean flag = false;
\par
System.out.println("+number = " + +number);
        // number is equal to 5.2 here.
\par
System.out.println("-number = " + -number);
        // number is equal to 5.2 here.
\par
// ++number is equivalent to number = number + 1
        System.out.println("number = " + ++number);
        // number is equal to 6.2 here.
\par
// -- number is equivalent to number = number - 1
        System.out.println("number = " + --number);
        // number is equal to 5.2 here.
\par
System.out.println("!flag = " + !flag);
        // flag is still false.
    }
}
\end{lstlisting}%
\lthtmlfigureZ
\lthtmlcheckvsize\clearpage}

{\newpage\clearpage
\lthtmlfigureA{lstlisting629}%
\begin{lstlisting}
+number = 5.2
-number = -5.2
number = 6.2
number = 5.2
!flag = true
\end{lstlisting}%
\lthtmlfigureZ
\lthtmlcheckvsize\clearpage}

\stepcounter{subsubsection}
{\newpage\clearpage
\lthtmlinlinemathA{tex2html_wrap_inline2998}%
$\lstinline{++}$%
\lthtmlindisplaymathZ
\lthtmlcheckvsize\clearpage}

{\newpage\clearpage
\lthtmlinlinemathA{tex2html_wrap_inline3000}%
$\lstinline{--}$%
\lthtmlindisplaymathZ
\lthtmlcheckvsize\clearpage}

{\newpage\clearpage
\lthtmlfigureA{lstlisting635}%
\begin{lstlisting}
int myInt = 5;
++myInt   // myInt becomes 6
myInt++   // myInt becomes 7
--myInt   // myInt becomes 6
myInt--   // myInt becomes 5
\end{lstlisting}%
\lthtmlfigureZ
\lthtmlcheckvsize\clearpage}

\stepcounter{subsubsection}
{\newpage\clearpage
\lthtmlfigureA{lstlisting638}%
\begin{lstlisting}
class UnaryOperator {
    public static void main(String[] args) {
\par
double number = 5.2;
\par
System.out.println(number++);
        System.out.println(number);
\par
System.out.println(++number);
        System.out.println(number);
    }
}
\end{lstlisting}%
\lthtmlfigureZ
\lthtmlcheckvsize\clearpage}

{\newpage\clearpage
\lthtmlfigureA{lstlisting641}%
\begin{lstlisting}
5.2
6.2
7.2
7.2
\end{lstlisting}%
\lthtmlfigureZ
\lthtmlcheckvsize\clearpage}

{\newpage\clearpage
\lthtmlinlinemathA{tex2html_wrap_inline3004}%
$\lstinline{System.out.println(number++);}$%
\lthtmlindisplaymathZ
\lthtmlcheckvsize\clearpage}

{\newpage\clearpage
\lthtmlinlinemathA{tex2html_wrap_inline3006}%
$\lstinline{System.out.println(number);}$%
\lthtmlindisplaymathZ
\lthtmlcheckvsize\clearpage}

{\newpage\clearpage
\lthtmlinlinemathA{tex2html_wrap_inline3008}%
$\texttt{System.out.println(++number);}$%
\lthtmlindisplaymathZ
\lthtmlcheckvsize\clearpage}

\stepcounter{subsection}
\stepcounter{subsubsection}
{\newpage\clearpage
\lthtmlfigureA{lstlisting657}%
\begin{lstlisting}
class RelationalOperator {
    public static void main(String[] args) {
\par
int number1 = 5, number2 = 6;
\par
if (number1 > number2) {
            System.out.println("number1 is greater than number2.");
        }
        else {
            System.out.println("number2 is greater than number1.");
        }
    }
}
\end{lstlisting}%
\lthtmlfigureZ
\lthtmlcheckvsize\clearpage}

\stepcounter{subsection}
{\newpage\clearpage
\lthtmlfigureA{lstlisting669}%
\begin{lstlisting}
class instanceofOperator {
    public static void main(String[] args) {
\par
String test = "asdf";
        boolean result;
\par
result = test instanceof String;
        System.out.println("Is test an object of String? " + result);
    }
}
\end{lstlisting}%
\lthtmlfigureZ
\lthtmlcheckvsize\clearpage}

\stepcounter{subsubsection}
{\newpage\clearpage
\lthtmlfigureA{lstlisting676}%
\begin{lstlisting}
result = objectName instanceof className;
\end{lstlisting}%
\lthtmlfigureZ
\lthtmlcheckvsize\clearpage}

\stepcounter{subsubsection}
{\newpage\clearpage
\lthtmlfigureA{lstlisting680}%
\begin{lstlisting}
class Main {
    public static void main (String[] args) {
        String name = "Programiz";
        Integer age = 22;
\par
System.out.println("Is name an instance of String: "+ (name instanceof String));
        System.out.println("Is age an instance of Integer: "+ (age instanceof Integer));
    }
}
\end{lstlisting}%
\lthtmlfigureZ
\lthtmlcheckvsize\clearpage}

{\newpage\clearpage
\lthtmlfigureA{lstlisting683}%
\begin{lstlisting}
Is name an instance of String: true
Is age an instance of Integer: true
\end{lstlisting}%
\lthtmlfigureZ
\lthtmlcheckvsize\clearpage}

\stepcounter{section}
{\newpage\clearpage
\lthtmlfigureA{lstlisting700}%
\begin{lstlisting}
String[] array = new String[100];
\end{lstlisting}%
\lthtmlfigureZ
\lthtmlcheckvsize\clearpage}

{\newpage\clearpage
\lthtmlfigureA{lstlisting702}%
\begin{lstlisting}
dataType[] arrayName;
\end{lstlisting}%
\lthtmlfigureZ
\lthtmlcheckvsize\clearpage}

{\newpage\clearpage
\lthtmlfigureA{lstlisting706}%
\begin{lstlisting}
double[] data;
\end{lstlisting}%
\lthtmlfigureZ
\lthtmlcheckvsize\clearpage}

{\newpage\clearpage
\lthtmlfigureA{lstlisting710}%
\begin{lstlisting}
// declare an array
double[] data;
\par
// allocate memory
data = new Double[10];
\end{lstlisting}%
\lthtmlfigureZ
\lthtmlcheckvsize\clearpage}

{\newpage\clearpage
\lthtmlfigureA{lstlisting712}%
\begin{lstlisting}
double[] data = new double[10];
\end{lstlisting}%
\lthtmlfigureZ
\lthtmlcheckvsize\clearpage}

\stepcounter{subsection}
{\newpage\clearpage
\lthtmlfigureA{lstlisting715}%
\begin{lstlisting}
//declare and initialize and array
int[] age = {12, 4, 5, 2, 5};
\end{lstlisting}%
\lthtmlfigureZ
\lthtmlcheckvsize\clearpage}

{\newpage\clearpage
\lthtmlfigureA{lstlisting718}%
\begin{lstlisting}
// declare an array
int[] age = new int[5];
\par
// initialize array
age[0] = 12;
age[1] = 4;
age[2] = 5;
..
\end{lstlisting}%
\lthtmlfigureZ
\lthtmlcheckvsize\clearpage}

\stepcounter{subsection}
{\newpage\clearpage
\lthtmlfigureA{lstlisting723}%
\begin{lstlisting}
// access array elements
array[index]
\end{lstlisting}%
\lthtmlfigureZ
\lthtmlcheckvsize\clearpage}

{\newpage\clearpage
\lthtmlfigureA{lstlisting725}%
\begin{lstlisting}
class Main {
 public static void main(String[] args) {
\par
// create an array
   int[] age = {12, 4, 5, 2, 5};
\par
// access each array elements
   System.out.println("Accessing Elements of Array:");
   System.out.println("First Element: " + age[0]);
   System.out.println("Second Element: " + age[1]);
   System.out.println("Third Element: " + age[2]);
   System.out.println("Fourth Element: " + age[3]);
   System.out.println("Fifth Element: " + age[4]);
 }
}
\end{lstlisting}%
\lthtmlfigureZ
\lthtmlcheckvsize\clearpage}

{\newpage\clearpage
\lthtmlfigureA{lstlisting728}%
\begin{lstlisting}
Accessing Elements of Array:
First Element: 12
Second Element: 4
Third Element: 5
Fourth Element: 2
Fifth Element: 5
\end{lstlisting}%
\lthtmlfigureZ
\lthtmlcheckvsize\clearpage}

\stepcounter{subsection}
{\newpage\clearpage
\lthtmlfigureA{lstlisting731}%
\begin{lstlisting}
class Main {
 public static void main(String[] args) {
\par
// create an array
   int[] age = {12, 4, 5};
\par
// loop through the array
   // using for loop
   System.out.println("Using for Loop:");
   for(int i = 0; i < age.length; i++) {
     System.out.println(age[i]);
   }
 }
}
\end{lstlisting}%
\lthtmlfigureZ
\lthtmlcheckvsize\clearpage}

{\newpage\clearpage
\lthtmlfigureA{lstlisting735}%
\begin{lstlisting}
Using for Loop:
12
4
5
\end{lstlisting}%
\lthtmlfigureZ
\lthtmlcheckvsize\clearpage}

{\newpage\clearpage
\lthtmlfigureA{lstlisting738}%
\begin{lstlisting}
class Main {
 public static void main(String[] args) {
\par
// create an array
   int[] age = {12, 4, 5};
\par
// loop through the array
   // using for loop
   System.out.println("Using for-each Loop:");
   for(int a : age) {
     System.out.println(a);
   }
 }
}
\end{lstlisting}%
\lthtmlfigureZ
\lthtmlcheckvsize\clearpage}

{\newpage\clearpage
\lthtmlfigureA{lstlisting742}%
\begin{lstlisting}
Using for-each Loop:
12
4
5
\end{lstlisting}%
\lthtmlfigureZ
\lthtmlcheckvsize\clearpage}

{\newpage\clearpage
\lthtmlfigureA{lstlisting744}%
\begin{lstlisting}
class Main {
 public static void main(String[] args) {
\par
int[] numbers = {2, -9, 0, 5, 12, -25, 22, 9, 8, 12};
   int sum = 0;
   Double average;
\par
// access all elements using for each loop
   // add each element in sum
   for (int number: numbers) {
     sum += number;
   }
\par
// get the total number of elements
   int arrayLength = numbers.length;
\par
// calculate the average
   // convert the average from int to double
   average =  ((double)sum / (double)arrayLength);
\par
System.out.println("Sum = " + sum);
   System.out.println("Average = " + average);
 }
}
\end{lstlisting}%
\lthtmlfigureZ
\lthtmlcheckvsize\clearpage}

{\newpage\clearpage
\lthtmlfigureA{lstlisting748}%
\begin{lstlisting}
Sum = 36
Average = 3.6
\end{lstlisting}%
\lthtmlfigureZ
\lthtmlcheckvsize\clearpage}

{\newpage\clearpage
\lthtmlfigureA{lstlisting753}%
\begin{lstlisting}
average = ((double)sum / (double)arrayLength);
\end{lstlisting}%
\lthtmlfigureZ
\lthtmlcheckvsize\clearpage}

\stepcounter{subsection}
{\newpage\clearpage
\lthtmlfigureA{lstlisting756}%
\begin{lstlisting}
double[][] matrix = {{1.2, 4.3, 4.0}, 
      {4.1, -1.1}
};
\end{lstlisting}%
\lthtmlfigureZ
\lthtmlcheckvsize\clearpage}

{\newpage\clearpage
\lthtmlfigureA{lstlisting760}%
\begin{lstlisting}
int[][] a = new int[3][4];
\end{lstlisting}%
\lthtmlfigureZ
\lthtmlcheckvsize\clearpage}

{\newpage\clearpage
\lthtmlfigureA{lstlisting763}%
\begin{lstlisting}
String[][][] data = new String[3][4][2];
\end{lstlisting}%
\lthtmlfigureZ
\lthtmlcheckvsize\clearpage}

\stepcounter{subsection}
{\newpage\clearpage
\lthtmlfigureA{lstlisting766}%
\begin{lstlisting}
int[][] a = {
      {1, 2, 3}, 
      {4, 5, 6, 9}, 
      {7}, 
};
\end{lstlisting}%
\lthtmlfigureZ
\lthtmlcheckvsize\clearpage}

{\newpage\clearpage
\lthtmlfigureA{lstlisting771}%
\begin{lstlisting}
class MultidimensionalArray {
    public static void main(String[] args) {
\par
// create a 2d array
        int[][] a = {
            {1, 2, 3}, 
            {4, 5, 6, 9}, 
            {7}, 
        };
\par
// calculate the length of each row
        System.out.println("Length of row 1: " + a[0].length);
        System.out.println("Length of row 2: " + a[1].length);
        System.out.println("Length of row 3: " + a[2].length);
    }
}
\end{lstlisting}%
\lthtmlfigureZ
\lthtmlcheckvsize\clearpage}

{\newpage\clearpage
\lthtmlfigureA{lstlisting776}%
\begin{lstlisting}
Length of row 1: 3
Length of row 2: 4
Length of row 3: 1
\end{lstlisting}%
\lthtmlfigureZ
\lthtmlcheckvsize\clearpage}

{\newpage\clearpage
\lthtmlfigureA{lstlisting780}%
\begin{lstlisting}
class MultidimensionalArray {
    public static void main(String[] args) {
\par
int[][] a = {
            {1, -2, 3}, 
            {-4, -5, 6, 9}, 
            {7}, 
        };
\par
for (int i = 0; i < a.length; ++i) {
            for(int j = 0; j < a[i].length; ++j) {
                System.out.println(a[i][j]);
            }
        }
    }
}
\end{lstlisting}%
\lthtmlfigureZ
\lthtmlcheckvsize\clearpage}

{\newpage\clearpage
\lthtmlfigureA{lstlisting786}%
\begin{lstlisting}
1
-2
3
-4
-5
6
9
7
\end{lstlisting}%
\lthtmlfigureZ
\lthtmlcheckvsize\clearpage}

{\newpage\clearpage
\lthtmlfigureA{lstlisting788}%
\begin{lstlisting}
class MultidimensionalArray {
    public static void main(String[] args) {
\par
// create a 2d array
        int[][] a = {
            {1, -2, 3}, 
            {-4, -5, 6, 9}, 
            {7}, 
        };
\par
// first for...each loop access the individual array
        // inside the 2d array
        for (int[] innerArray: a) {
            // second for...each loop access each element inside the row
            for(int data: innerArray) {
                System.out.println(data);
            }
        }
    }
}
\end{lstlisting}%
\lthtmlfigureZ
\lthtmlcheckvsize\clearpage}

{\newpage\clearpage
\lthtmlfigureA{lstlisting798}%
\begin{lstlisting}
// test is a 3d array
int[][][] test = {
        {
          {1, -2, 3}, 
          {2, 3, 4}
        }, 
        { 
          {-4, -5, 6, 9}, 
          {1}, 
          {2, 3}
        } 
};
\end{lstlisting}%
\lthtmlfigureZ
\lthtmlcheckvsize\clearpage}

{\newpage\clearpage
\lthtmlfigureA{lstlisting805}%
\begin{lstlisting}
class ThreeArray {
    public static void main(String[] args) {
\par
// create a 3d array
        int[][][] test = {
            {
              {1, -2, 3}, 
              {2, 3, 4}
            }, 
            { 
              {-4, -5, 6, 9}, 
              {1}, 
              {2, 3}
            } 
        };
\par
// for..each loop to iterate through elements of 3d array
        for (int[][] array2D: test) {
            for (int[] array1D: array2D) {
                for(int item: array1D) {
                    System.out.println(item);
                }
            }
        }
    }
}
\end{lstlisting}%
\lthtmlfigureZ
\lthtmlcheckvsize\clearpage}

{\newpage\clearpage
\lthtmlfigureA{lstlisting813}%
\begin{lstlisting}
1
-2
3
2
3
4
-4
-5
6
9
1
2
3
\end{lstlisting}%
\lthtmlfigureZ
\lthtmlcheckvsize\clearpage}

\stepcounter{subsection}
{\newpage\clearpage
\lthtmlfigureA{lstlisting816}%
\begin{lstlisting}
class Main {
    public static void main(String[] args) {
\par
int [] numbers = {1, 2, 3, 4, 5, 6};
        int [] positiveNumbers = numbers;    // copying arrays
\par
for (int number: positiveNumbers) {
            System.out.print(number + ", ");
        }
    }
}
\end{lstlisting}%
\lthtmlfigureZ
\lthtmlcheckvsize\clearpage}

{\newpage\clearpage
\lthtmlfigureA{lstlisting824}%
\begin{lstlisting}
class Main {
    public static void main(String[] args) {
\par
int [] numbers = {1, 2, 3, 4, 5, 6};
        int [] positiveNumbers = numbers;    // copying arrays
\par
// change value of first array
        numbers[0] = -1;
\par
// printing the second array
        for (int number: positiveNumbers) {
            System.out.print(number + ", ");
        }
    }
}
\end{lstlisting}%
\lthtmlfigureZ
\lthtmlcheckvsize\clearpage}

\stepcounter{subsection}
{\newpage\clearpage
\lthtmlfigureA{lstlisting832}%
\begin{lstlisting}
import java.util.Arrays;
\par
class Main {
    public static void main(String[] args) {
\par
int [] source = {1, 2, 3, 4, 5, 6};
        int [] destination = new int[6];
\par
// iterate and copy elements from source to destination
        for (int i = 0; i < source.length; ++i) {
            destination[i] = source[i];
        }
\par
// converting array to string
        System.out.println(Arrays.toString(destination));
    }
}
\end{lstlisting}%
\lthtmlfigureZ
\lthtmlcheckvsize\clearpage}

\stepcounter{subsection}
{\newpage\clearpage
\lthtmlfigureA{lstlisting844}%
\begin{lstlisting}
arraycopy(Object src, int srcPos,Object dest, int destPos, int length)
\end{lstlisting}%
\lthtmlfigureZ
\lthtmlcheckvsize\clearpage}

{\newpage\clearpage
\lthtmlfigureA{lstlisting853}%
\begin{lstlisting}
// To use Arrays.toString() method
import java.util.Arrays;
\par
class Main {
    public static void main(String[] args) {
        int[] n1 = {2, 3, 12, 4, 12, -2};
\par
int[] n3 = new int[5];
\par
// Creating n2 array of having length of n1 array
        int[] n2 = new int[n1.length];
\par
// copying entire n1 array to n2
        System.arraycopy(n1, 0, n2, 0, n1.length);
        System.out.println("n2 = " + Arrays.toString(n2));  
\par
// copying elements from index 2 on n1 array
        // copying element to index 1 of n3 array
        // 2 elements will be copied
        System.arraycopy(n1, 2, n3, 1, 2);
        System.out.println("n3 = " + Arrays.toString(n3));  
    }
}
\end{lstlisting}%
\lthtmlfigureZ
\lthtmlcheckvsize\clearpage}

{\newpage\clearpage
\lthtmlfigureA{lstlisting856}%
\begin{lstlisting}
n2 = [2, 3, 12, 4, 12, -2]
n3 = [0, 12, 4, 0, 0]
\end{lstlisting}%
\lthtmlfigureZ
\lthtmlcheckvsize\clearpage}

\stepcounter{subsection}
{\newpage\clearpage
\lthtmlfigureA{lstlisting869}%
\begin{lstlisting}
// To use toString() and copyOfRange() method
import java.util.Arrays;
\par
class ArraysCopy {
    public static void main(String[] args) {
\par
int[] source = {2, 3, 12, 4, 12, -2};
\par
// copying entire source array to destination
        int[] destination1 = Arrays.copyOfRange(source, 0, source.length);      
        System.out.println("destination1 = " + Arrays.toString(destination1)); 
\par
// copying from index 2 to 5 (5 is not included) 
        int[] destination2 = Arrays.copyOfRange(source, 2, 5); 
        System.out.println("destination2 = " + Arrays.toString(destination2));   
    }
}
\end{lstlisting}%
\lthtmlfigureZ
\lthtmlcheckvsize\clearpage}

{\newpage\clearpage
\lthtmlfigureA{lstlisting872}%
\begin{lstlisting}
destination1 = [2, 3, 12, 4, 12, -2]
destination2 = [12, 4, 12]
\end{lstlisting}%
\lthtmlfigureZ
\lthtmlcheckvsize\clearpage}

{\newpage\clearpage
\lthtmlfigureA{lstlisting874}%
\begin{lstlisting}
int[] destination1 = Arrays.copyOfRange(source, 0, source.length);
\end{lstlisting}%
\lthtmlfigureZ
\lthtmlcheckvsize\clearpage}

\stepcounter{subsection}
{\newpage\clearpage
\lthtmlfigureA{lstlisting882}%
\begin{lstlisting}
import java.util.Arrays;
\par
class Main {
    public static void main(String[] args) {
\par
int[][] source = {
              {1, 2, 3, 4}, 
              {5, 6},
              {0, 2, 42, -4, 5}
              };
\par
int[][] destination = new int[source.length][];
\par
for (int i = 0; i < destination.length; ++i) {
\par
// allocating space for each row of destination array
            destination[i] = new int[source[i].length];
\par
for (int j = 0; j < destination[i].length; ++j) {
                destination[i][j] = source[i][j];
            }
        }
\par
// displaying destination array
        System.out.println(Arrays.deepToString(destination));  
\par
}
}
\end{lstlisting}%
\lthtmlfigureZ
\lthtmlcheckvsize\clearpage}

{\newpage\clearpage
\lthtmlfigureA{lstlisting888}%
\begin{lstlisting}
[[1, 2, 3, 4], [5, 6], [0, 2, 42, -4, 5]]
\end{lstlisting}%
\lthtmlfigureZ
\lthtmlcheckvsize\clearpage}

{\newpage\clearpage
\lthtmlfigureA{lstlisting890}%
\begin{lstlisting}
System.out.println(Arrays.deepToString(destination);
\end{lstlisting}%
\lthtmlfigureZ
\lthtmlcheckvsize\clearpage}

\stepcounter{subsection}
{\newpage\clearpage
\lthtmlfigureA{lstlisting895}%
\begin{lstlisting}
import java.util.Arrays;
\par
class Main {
    public static void main(String[] args) {
\par
int[][] source = {
              {1, 2, 3, 4}, 
              {5, 6},
              {0, 2, 42, -4, 5}
              };
\par
int[][] destination = new int[source.length][];
\par
for (int i = 0; i < source.length; ++i) {
\par
// allocating space for each row of destination array
             destination[i] = new int[source[i].length];
             System.arraycopy(source[i], 0, destination[i], 0, destination[i].length);
        }
\par
// displaying destination array
        System.out.println(Arrays.deepToString(destination));      
    }
}
\end{lstlisting}%
\lthtmlfigureZ
\lthtmlcheckvsize\clearpage}

\stepcounter{section}
\stepcounter{subsection}
\stepcounter{subsection}
\stepcounter{subsection}
{\newpage\clearpage
\lthtmlfigureA{lstlisting944}%
\begin{lstlisting}
// The Collections framework is defined in the java.util package
import java.util.ArrayList;
\par
class Main {
    public static void main(String[] args){
        ArrayList<String> animals = new ArrayList<>();
        // Add elements
        animals.add("Dog");
        animals.add("Cat");
        animals.add("Horse");
\par
System.out.println("ArrayList: " + animals);
    }
}
\end{lstlisting}%
\lthtmlfigureZ
\lthtmlcheckvsize\clearpage}

\stepcounter{subsection}
\stepcounter{subsection}
\stepcounter{subsubsection}
{\newpage\clearpage
\lthtmlfigureA{lstlisting968}%
\begin{lstlisting}
// ArrayList implementation of List
List<String> list1 = new ArrayList<>();
\par
// LinkedList implementation of List
List<String> list2 = new LinkedList<>();
\end{lstlisting}%
\lthtmlfigureZ
\lthtmlcheckvsize\clearpage}

\stepcounter{subsubsection}
\stepcounter{subsubsection}
{\newpage\clearpage
\lthtmlfigureA{lstlisting996}%
\begin{lstlisting}
import java.util.List;
import java.util.ArrayList;
\par
class Main {
\par
public static void main(String[] args) {
        // Creating list using the ArrayList class
        List<Integer> numbers = new ArrayList<>();
\par
// Add elements to the list
        numbers.add(1);
        numbers.add(2);
        numbers.add(3);
        System.out.println("List: " + numbers);
\par
// Access element from the list
        int number = numbers.get(2);
        System.out.println("Accessed Element: " + number);
\par
// Remove element from the list
        int removedNumber = numbers.remove(1);
        System.out.println("Removed Element: " + removedNumber);
    }
}
\end{lstlisting}%
\lthtmlfigureZ
\lthtmlcheckvsize\clearpage}

{\newpage\clearpage
\lthtmlfigureA{lstlisting999}%
\begin{lstlisting}
List: [1, 2, 3]
Accessed Element: 3
Removed Element: 2
\end{lstlisting}%
\lthtmlfigureZ
\lthtmlcheckvsize\clearpage}

\stepcounter{subsubsection}
{\newpage\clearpage
\lthtmlfigureA{lstlisting1002}%
\begin{lstlisting}
import java.util.List;
import java.util.LinkedList;
\par
class Main {
\par
public static void main(String[] args) {
        // Creating list using the LinkedList class
        List<Integer> numbers = new LinkedList<>();
\par
// Add elements to the list
        numbers.add(1);
        numbers.add(2);
        numbers.add(3);
        System.out.println("List: " + numbers);
\par
// Access element from the list
        int number = numbers.get(2);
        System.out.println("Accessed Element: " + number);
\par
// Using the indexOf() method
        int index = numbers.indexOf(2);
        System.out.println("Position of 3 is " + index);
\par
// Remove element from the list
        int removedNumber = numbers.remove(1);
        System.out.println("Removed Element: " + removedNumber);
    }
}
\end{lstlisting}%
\lthtmlfigureZ
\lthtmlcheckvsize\clearpage}

{\newpage\clearpage
\lthtmlfigureA{lstlisting1005}%
\begin{lstlisting}
List: [1, 2, 3]
Accessed Element: 3
Position of 3 is 1
Removed Element: 2
\end{lstlisting}%
\lthtmlfigureZ
\lthtmlcheckvsize\clearpage}

\stepcounter{subsubsection}
\stepcounter{subsubsection}
{\newpage\clearpage
\lthtmlfigureA{lstlisting1017}%
\begin{lstlisting}
import java.util.ArrayList;
\par
class Main {
    public static void main(String[] args){
        ArrayList<String> animals = new ArrayList<>();
\par
// Add elements
        animals.add(0,"Dog");
        animals.add(1,"Cat");
        animals.add(2,"Horse");
        System.out.println("ArrayList: " + animals);
    }
}
\end{lstlisting}%
\lthtmlfigureZ
\lthtmlcheckvsize\clearpage}

{\newpage\clearpage
\lthtmlfigureA{lstlisting1020}%
\begin{lstlisting}
ArrayList: [Dog, Cat, Horse]
\end{lstlisting}%
\lthtmlfigureZ
\lthtmlcheckvsize\clearpage}

\stepcounter{subsubsection}
{\newpage\clearpage
\lthtmlfigureA{lstlisting1024}%
\begin{lstlisting}
import java.util.ArrayList;
\par
class Main {
    public static void main(String[] args){
        ArrayList<String> mammals = new ArrayList<>();
        mammals.add("Dog");
        mammals.add("Cat");
        mammals.add("Horse");
        System.out.println("Mammals: " + mammals);
\par
ArrayList<String> animals = new ArrayList<>();
        animals.add("Crocodile");
\par
// Add all elements of mammals in animals
        animals.addAll(mammals);
        System.out.println("Animals: " + animals);
    }
}
\end{lstlisting}%
\lthtmlfigureZ
\lthtmlcheckvsize\clearpage}

{\newpage\clearpage
\lthtmlfigureA{lstlisting1027}%
\begin{lstlisting}
Mammals: [Dog, Cat, Horse]
Animals: [Crocodile, Dog, Cat, Horse]
\end{lstlisting}%
\lthtmlfigureZ
\lthtmlcheckvsize\clearpage}

\stepcounter{subsubsection}
{\newpage\clearpage
\lthtmlfigureA{lstlisting1034}%
\begin{lstlisting}
import java.util.ArrayList;
import java.util.Arrays;
\par
class Main {
    public static void main(String[] args) {
        // Creating an array list
        ArrayList<String> animals = new ArrayList<>(Arrays.asList("Cat", "Cow", "Dog"));
        System.out.println("ArrayList: " + animals);
\par
// Access elements of the array list
        String element = animals.get(1);
        System.out.println("Accessed Element: " + element);
    }
}
\end{lstlisting}%
\lthtmlfigureZ
\lthtmlcheckvsize\clearpage}

{\newpage\clearpage
\lthtmlfigureA{lstlisting1037}%
\begin{lstlisting}
ArrayList: [Cat, Cow, Dog]
Accessed Elemenet: Cow
\end{lstlisting}%
\lthtmlfigureZ
\lthtmlcheckvsize\clearpage}

{\newpage\clearpage
\lthtmlfigureA{lstlisting1039}%
\begin{lstlisting}
new ArrayList<>(Arrays.asList(("Cat", "Cow", "Dog"));
\end{lstlisting}%
\lthtmlfigureZ
\lthtmlcheckvsize\clearpage}

\stepcounter{subsection}
\stepcounter{subsubsection}
{\newpage\clearpage
\lthtmlfigureA{lstlisting1047}%
\begin{lstlisting}
import java.util.ArrayList;
\par
class Main {
    public static void main(String[] args) {
        ArrayList<String> animals= new ArrayList<>();
\par
// Add elements in the array list
        animals.add("Dog");
        animals.add("Horse");
        animals.add("Cat");
        System.out.println("ArrayList: " + animals);
\par
// Get the element from the array list
        String str = animals.get(0);
        System.out.print("Element at index 0: " + str);
    }
}
\end{lstlisting}%
\lthtmlfigureZ
\lthtmlcheckvsize\clearpage}

{\newpage\clearpage
\lthtmlfigureA{lstlisting1050}%
\begin{lstlisting}
ArrayList: [Dog, Horse, Cat]
Element at index 0: Dog
\end{lstlisting}%
\lthtmlfigureZ
\lthtmlcheckvsize\clearpage}

\stepcounter{subsubsection}
{\newpage\clearpage
\lthtmlfigureA{lstlisting1055}%
\begin{lstlisting}
import java.util.ArrayList;
import java.util.Iterator;
\par
class Main {
    public static void main(String[] args){
        ArrayList<String> animals = new ArrayList<>();
\par
// Add elements in the array list
        animals.add("Dog");
        animals.add("Cat");
        animals.add("Horse");
        animals.add("Zebra");
\par
// Create an object of Iterator
        Iterator<String> iterate = animals.iterator();
        System.out.print("ArrayList: ");
\par
// Use methods of Iterator to access elements
        while(iterate.hasNext()){
            System.out.print(iterate.next());
            System.out.print(", ");
        }
    }
}
\end{lstlisting}%
\lthtmlfigureZ
\lthtmlcheckvsize\clearpage}

\stepcounter{subsubsection}
{\newpage\clearpage
\lthtmlfigureA{lstlisting1063}%
\begin{lstlisting}
import java.util.ArrayList;
\par
class Main {
    public static void main(String[] args) {
        ArrayList<String> animals= new ArrayList<>();
        // Add elements in the array list
        animals.add("Dog");
        animals.add("Cat");
        animals.add("Horse");
        System.out.println("ArrayList: " + animals);
\par
// Change the element of the array list
        animals.set(2, "Zebra");
        System.out.println("Modified ArrayList: " + animals);
    }
}
\end{lstlisting}%
\lthtmlfigureZ
\lthtmlcheckvsize\clearpage}

{\newpage\clearpage
\lthtmlfigureA{lstlisting1066}%
\begin{lstlisting}
ArrayList: [Dog, Cat, Horse]
Modified ArrayList: [Dog, Cat, Zebra]
\end{lstlisting}%
\lthtmlfigureZ
\lthtmlcheckvsize\clearpage}

\stepcounter{subsection}
\stepcounter{subsubsection}
{\newpage\clearpage
\lthtmlfigureA{lstlisting1071}%
\begin{lstlisting}
import java.util.ArrayList;
\par
class Main {
    public static void main(String[] args) {
        ArrayList<String> animals = new ArrayList<>();
\par
// Add elements in the array list
        animals.add("Dog");
        animals.add("Cat");
        animals.add("Horse");
        System.out.println("Initial ArrayList: " + animals);
\par
// Remove element from index 2
        String str = animals.remove(2);
        System.out.println("Final ArrayList: " + animals);
        System. out.println("Removed Element: " + str);
    }
}
\end{lstlisting}%
\lthtmlfigureZ
\lthtmlcheckvsize\clearpage}

{\newpage\clearpage
\lthtmlfigureA{lstlisting1074}%
\begin{lstlisting}
Initial ArrayList: [Dog, Cat, Horse]
Final ArrayList: [Dog, Cat]
Removed Element: Horse
\end{lstlisting}%
\lthtmlfigureZ
\lthtmlcheckvsize\clearpage}

\stepcounter{subsubsection}
{\newpage\clearpage
\lthtmlfigureA{lstlisting1078}%
\begin{lstlisting}
import java.util.ArrayList;
\par
class Main {
    public static void main(String[] args) {
        ArrayList<String> animals = new ArrayList<>();
\par
// Add elements in the ArrayList
        animals.add("Dog");
        animals.add("Cat");
        animals.add("Horse");
        System.out.println("Initial ArrayList: " + animals);
\par
// Remove all the elements
        animals.removeAll(animals);
        System.out.println("Final ArrayList: " + animals);
    }
}
\end{lstlisting}%
\lthtmlfigureZ
\lthtmlcheckvsize\clearpage}

{\newpage\clearpage
\lthtmlfigureA{lstlisting1081}%
\begin{lstlisting}
Initial ArrayList: [Dog, Cat, Horse]
Final ArrayList: []
\end{lstlisting}%
\lthtmlfigureZ
\lthtmlcheckvsize\clearpage}

\stepcounter{subsubsection}
{\newpage\clearpage
\lthtmlfigureA{lstlisting1085}%
\begin{lstlisting}
import java.util.ArrayList;
\par
class Main {
    public static void main(String[] args) {
        ArrayList<String> animals= new ArrayList<>();
\par
// Add elements in the array list
        animals.add("Dog");
        animals.add("Cat");
        animals.add("Horse");
        System.out.println("Initial ArrayList: " + animals);
\par
// Remove all the elements
        animals.clear();
        System.out.println("Final ArrayList: " + animals);
    }
}
\end{lstlisting}%
\lthtmlfigureZ
\lthtmlcheckvsize\clearpage}

\stepcounter{subsection}
\stepcounter{subsubsection}
{\newpage\clearpage
\lthtmlfigureA{lstlisting1092}%
\begin{lstlisting}
import java.util.ArrayList;
\par
class Main {
    public static void main(String[] args) {
        // Creating an array list
        ArrayList<String> animals = new ArrayList<>();
        animals.add("Cow");
        animals.add("Cat");
        animals.add("Dog");
        System.out.println("ArrayList: " + animals);
\par
// Using for loop
        System.out.println("Accessing individual elements: ");
\par
for(int i = 0; i < animals.size(); i++) {
            System.out.print(animals.get(i));
            System.out.print(", ");
        }
    }
}
\end{lstlisting}%
\lthtmlfigureZ
\lthtmlcheckvsize\clearpage}

{\newpage\clearpage
\lthtmlfigureA{lstlisting1095}%
\begin{lstlisting}
ArrayList: [Cow, Cat, Dog]
Accessing individual elements:
Cow, Cat, Dog,
\end{lstlisting}%
\lthtmlfigureZ
\lthtmlcheckvsize\clearpage}

\stepcounter{subsubsection}
{\newpage\clearpage
\lthtmlfigureA{lstlisting1098}%
\begin{lstlisting}
import java.util.ArrayList;
\par
class Main {
    public static void main(String[] args) {
        // Creating an array list
        ArrayList<String> animals = new ArrayList<>();
        animals.add("Cow");
        animals.add("Cat");
        animals.add("Dog");
        System.out.println("ArrayList: " + animals);
\par
// Using forEach loop
        System.out.println("Accessing individual elements:  ");
        for(String animal : animals) {
            System.out.print(animal);
            System.out.print(", ");
        }
    }
}
\end{lstlisting}%
\lthtmlfigureZ
\lthtmlcheckvsize\clearpage}

\stepcounter{subsubsection}
{\newpage\clearpage
\lthtmlfigureA{lstlisting1105}%
\begin{lstlisting}
import java.util.ArrayList;
\par
class Main {
    public static void main(String[] args) {
        ArrayList<String> animals= new ArrayList<>();
\par
// Adding elements in the arrayList
        animals.add("Dog");
        animals.add("Horse");
        animals.add("Cat");
        System.out.println("ArrayList: " + animals);
\par
// getting the size of the arrayList
        System.out.println("Size: " + animals.size());
    }
}
\end{lstlisting}%
\lthtmlfigureZ
\lthtmlcheckvsize\clearpage}

{\newpage\clearpage
\lthtmlfigureA{lstlisting1108}%
\begin{lstlisting}
ArrayList: [Dog, Horse, Cat]
Size: 3
\end{lstlisting}%
\lthtmlfigureZ
\lthtmlcheckvsize\clearpage}

\stepcounter{subsubsection}
{\newpage\clearpage
\lthtmlfigureA{lstlisting1114}%
\begin{lstlisting}
import java.util.ArrayList;
import java.util.Collections;
\par
class Main {
    public static void main(String[] args){
        ArrayList<String> animals= new ArrayList<>();
\par
// Add elements in the array list
        animals.add("Horse");
        animals.add("Zebra");
        animals.add("Dog");
        animals.add("Cat");
\par
System.out.println("Unsorted ArrayList: " + animals);
\par
// Sort the array list
        Collections.sort(animals);
        System.out.println("Sorted ArrayList: " + animals);
    }
}
\end{lstlisting}%
\lthtmlfigureZ
\lthtmlcheckvsize\clearpage}

{\newpage\clearpage
\lthtmlfigureA{lstlisting1117}%
\begin{lstlisting}
Unsorted ArrayList: [Horse, Zebra, Dog, Cat]
Sorted ArrayList: [Cat, Dog, Horse, Zebra]
\end{lstlisting}%
\lthtmlfigureZ
\lthtmlcheckvsize\clearpage}

\stepcounter{subsubsection}
{\newpage\clearpage
\lthtmlfigureA{lstlisting1121}%
\begin{lstlisting}
import java.util.ArrayList;
\par
class Main {
    public static void main(String[] args) {
        ArrayList<String> animals= new ArrayList<>();
\par
// Add elements in the array list
        animals.add("Dog");
        animals.add("Cat");
        animals.add("Horse");
        System.out.println("ArrayList: " + animals);
\par
// Create a new array of String type
        String[] arr = new String[animals.size()];
\par
// Convert ArrayList into an array
        animals.toArray(arr);
        System.out.print("Array: ");
        for(String item:arr) {
            System.out.print(item+", ");
        }
    }
}
\end{lstlisting}%
\lthtmlfigureZ
\lthtmlcheckvsize\clearpage}

{\newpage\clearpage
\lthtmlfigureA{lstlisting1124}%
\begin{lstlisting}
ArrayList: [Dog, Cat, Horse]
Array: Dog, Cat, Horse,
\end{lstlisting}%
\lthtmlfigureZ
\lthtmlcheckvsize\clearpage}

\stepcounter{subsubsection}
{\newpage\clearpage
\lthtmlfigureA{lstlisting1131}%
\begin{lstlisting}
import java.util.ArrayList;
import java.util.Arrays;
\par
class Main {
    public static void main(String[] args) {
        // Create an array of String type
        String[] arr = {"Dog", "Cat", "Horse"};
        System.out.print("Array: ");
\par
// Print array
        for(String str: arr) {
            System.out.print(str);
            System.out.print(" ");
        }
\par
// Create an ArrayList from an array
        ArrayList<String> animals = new ArrayList<>(Arrays.asList(arr));
        System.out.println("\nArrayList: " + animals);
    }
}
\end{lstlisting}%
\lthtmlfigureZ
\lthtmlcheckvsize\clearpage}

{\newpage\clearpage
\lthtmlfigureA{lstlisting1135}%
\begin{lstlisting}
Array: Dog, Cat, Horse
ArrayList: [Dog, Cat, Horse]
\par
\end{lstlisting}%
\lthtmlfigureZ
\lthtmlcheckvsize\clearpage}

\stepcounter{subsubsection}
{\newpage\clearpage
\lthtmlfigureA{lstlisting1142}%
\begin{lstlisting}
import java.util.ArrayList;
\par
class Main {
    public static void main(String[] args) {
        ArrayList<String> animals = new ArrayList<>();
\par
// Add elements in the ArrayList
        animals.add("Dog");
        animals.add("Cat");
        animals.add("Horse");
        System.out.println("ArrayList: " + animals);
\par
// Convert ArrayList into an String
        String str = animals.toString();
        System.out.println("String: " + str);
    }
}
\end{lstlisting}%
\lthtmlfigureZ
\lthtmlcheckvsize\clearpage}

{\newpage\clearpage
\lthtmlfigureA{lstlisting1145}%
\begin{lstlisting}
ArrayList: [Dog, Cat, Horse]
String: [Dog, Cat, Horse]
\end{lstlisting}%
\lthtmlfigureZ
\lthtmlcheckvsize\clearpage}

\stepcounter{subsection}
\stepcounter{subsubsection}
\stepcounter{subsubsection}
{\newpage\clearpage
\lthtmlfigureA{lstlisting1174}%
\begin{lstlisting}
Vector<Type> vector = new Vector<>();
\end{lstlisting}%
\lthtmlfigureZ
\lthtmlcheckvsize\clearpage}

{\newpage\clearpage
\lthtmlfigureA{lstlisting1176}%
\begin{lstlisting}
// create Integer type linked list
Vector<Integer> vector= new Vector<>();
\par
// create String type linked list
Vector<String> vector= new Vector<>();
\end{lstlisting}%
\lthtmlfigureZ
\lthtmlcheckvsize\clearpage}

\stepcounter{subsubsection}
{\newpage\clearpage
\lthtmlfigureA{lstlisting1189}%
\begin{lstlisting}
import java.util.Vector;
\par
class Main {
    public static void main(String[] args) {
        Vector<String> mammals= new Vector<>();
\par
// Using the add() method
        mammals.add("Dog");
        mammals.add("Horse");
\par
// Using index number
        mammals.add(2, "Cat");
        System.out.println("Vector: " + mammals);
\par
// Using addAll()
        Vector<String> animals = new Vector<>();
        animals.add("Crocodile");
\par
animals.addAll(mammals);
        System.out.println("New Vector: " + animals);
    }
}
\end{lstlisting}%
\lthtmlfigureZ
\lthtmlcheckvsize\clearpage}

{\newpage\clearpage
\lthtmlfigureA{lstlisting1192}%
\begin{lstlisting}
Vector: [Dog, Horse, Cat]
New Vector: [Crocodile, Dog, Horse, Cat]
\end{lstlisting}%
\lthtmlfigureZ
\lthtmlcheckvsize\clearpage}

{\newpage\clearpage
\lthtmlfigureA{lstlisting1199}%
\begin{lstlisting}
import java.util.Iterator;
import java.util.Vector;
\par
class Main {
    public static void main(String[] args) {
        Vector<String> animals= new Vector<>();
        animals.add("Dog");
        animals.add("Horse");
        animals.add("Cat");
\par
// Using get()
        String element = animals.get(2);
        System.out.println("Element at index 2: " + element);
\par
// Using iterator()
        Iterator<String> iterate = animals.iterator();
        System.out.print("Vector: ");
        while(iterate.hasNext()) {
            System.out.print(iterate.next());
            System.out.print(", ");
        }
    }
}
\end{lstlisting}%
\lthtmlfigureZ
\lthtmlcheckvsize\clearpage}

{\newpage\clearpage
\lthtmlfigureA{lstlisting1202}%
\begin{lstlisting}
Element at index 2: Cat
Vector: Dog, Horse, Cat,
\end{lstlisting}%
\lthtmlfigureZ
\lthtmlcheckvsize\clearpage}

{\newpage\clearpage
\lthtmlfigureA{lstlisting1210}%
\begin{lstlisting}
import java.util.Vector;
\par
class Main {
    public static void main(String[] args) {
        Vector<String> animals= new Vector<>();
        animals.add("Dog");
        animals.add("Horse");
        animals.add("Cat");
\par
System.out.println("Initial Vector: " + animals);
\par
// Using remove()
        String element = animals.remove(1);
        System.out.println("Removed Element: " + element);
        System.out.println("New Vector: " + animals);
\par
// Using clear()
        animals.clear();
        System.out.println("Vector after clear(): " + animals);
    }
}
\end{lstlisting}%
\lthtmlfigureZ
\lthtmlcheckvsize\clearpage}

{\newpage\clearpage
\lthtmlfigureA{lstlisting1213}%
\begin{lstlisting}
Initial Vector: [Dog, Horse, Cat]
Removed Element: Horse
New Vector: [Dog, Cat]
Vector after clear(): []
\end{lstlisting}%
\lthtmlfigureZ
\lthtmlcheckvsize\clearpage}

\stepcounter{subsubsection}
\stepcounter{subsection}
\stepcounter{subsubsection}
{\newpage\clearpage
\lthtmlfigureA{lstlisting1237}%
\begin{lstlisting}
Stack<Type> stacks = new Stack<>();
\end{lstlisting}%
\lthtmlfigureZ
\lthtmlcheckvsize\clearpage}

{\newpage\clearpage
\lthtmlfigureA{lstlisting1239}%
\begin{lstlisting}
// Create Integer type stack
Stack<Integer> stacks = new Stack<>();
\par
// Create String type stack
Stack<String> stacks = new Stack<>();
\end{lstlisting}%
\lthtmlfigureZ
\lthtmlcheckvsize\clearpage}

\stepcounter{subsubsection}
\stepcounter{subsubsection}
{\newpage\clearpage
\lthtmlfigureA{lstlisting1251}%
\begin{lstlisting}
import java.util.Stack;
\par
class Main {
    public static void main(String[] args) {
        Stack<String> animals= new Stack<>();
\par
// Add elements to Stack
        animals.push("Dog");
        animals.push("Horse");
        animals.push("Cat");
\par
System.out.println("Stack: " + animals);
    }
}
\end{lstlisting}%
\lthtmlfigureZ
\lthtmlcheckvsize\clearpage}

{\newpage\clearpage
\lthtmlfigureA{lstlisting1254}%
\begin{lstlisting}
Stack: [Dog, Horse, Cat]
\end{lstlisting}%
\lthtmlfigureZ
\lthtmlcheckvsize\clearpage}

\stepcounter{subsubsection}
{\newpage\clearpage
\lthtmlfigureA{lstlisting1258}%
\begin{lstlisting}
import java.util.Stack;
\par
class Main {
    public static void main(String[] args) {
        Stack<String> animals= new Stack<>();
\par
// Add elements to Stack
        animals.push("Dog");
        animals.push("Horse");
        animals.push("Cat");
        System.out.println("Initial Stack: " + animals);
\par
// Remove element stacks
        String element = animals.pop();
        System.out.println("Removed Element: " + element);
    }
}
\end{lstlisting}%
\lthtmlfigureZ
\lthtmlcheckvsize\clearpage}

\stepcounter{subsubsection}
{\newpage\clearpage
\lthtmlfigureA{lstlisting1263}%
\begin{lstlisting}
import java.util.Stack;
\par
class Main {
    public static void main(String[] args) {
        Stack<String> animals= new Stack<>();
\par
// Add elements to Stack
        animals.push("Dog");
        animals.push("Horse");
        animals.push("Cat");
        System.out.println("Stack: " + animals);
\par
// Access element from the top
        String element = animals.peek();
        System.out.println("Element at top: " + element);
\par
}
}
\end{lstlisting}%
\lthtmlfigureZ
\lthtmlcheckvsize\clearpage}

{\newpage\clearpage
\lthtmlfigureA{lstlisting1266}%
\begin{lstlisting}
Stack: [Dog, Horse, Cat]
Element at top: Cat
\end{lstlisting}%
\lthtmlfigureZ
\lthtmlcheckvsize\clearpage}

\stepcounter{subsubsection}
{\newpage\clearpage
\lthtmlfigureA{lstlisting1270}%
\begin{lstlisting}
import java.util.Stack;
\par
class Main {
    public static void main(String[] args) {
        Stack<String> animals= new Stack<>();
\par
// Add elements to Stack
        animals.push("Dog");
        animals.push("Horse");
        animals.push("Cat");
        System.out.println("Stack: " + animals);
\par
// Search an element
        int position = animals.search("Horse");
        System.out.println("Position of Horse: " + position);
    }
}
\end{lstlisting}%
\lthtmlfigureZ
\lthtmlcheckvsize\clearpage}

{\newpage\clearpage
\lthtmlfigureA{lstlisting1273}%
\begin{lstlisting}
Stack: [Dog, Horse, Cat]
Position of Horse: 2
\end{lstlisting}%
\lthtmlfigureZ
\lthtmlcheckvsize\clearpage}

\stepcounter{subsubsection}
{\newpage\clearpage
\lthtmlfigureA{lstlisting1277}%
\begin{lstlisting}
import java.util.Stack;
\par
class Main {
    public static void main(String[] args) {
        Stack<String> animals= new Stack<>();
\par
// Add elements to Stack
        animals.push("Dog");
        animals.push("Horse");
        animals.push("Cat");
        System.out.println("Stack: " + animals);
\par
// Check if stack is empty
        boolean result = animals.empty();
        System.out.println("Is the stack empty? " + result);
    }
}
\end{lstlisting}%
\lthtmlfigureZ
\lthtmlcheckvsize\clearpage}

{\newpage\clearpage
\lthtmlfigureA{lstlisting1280}%
\begin{lstlisting}
Stack: [Dog, Horse, Cat]
Is the stack empty? false
\end{lstlisting}%
\lthtmlfigureZ
\lthtmlcheckvsize\clearpage}

\stepcounter{section}
\stepcounter{subsection}
\stepcounter{subsection}
{\newpage\clearpage
\lthtmlfigureA{lstlisting1299}%
\begin{lstlisting}
// LinkedList implementation of Queue
Queue<String> animal1 = new LinkedList<>();
\par
// Array implementation of Queue
Queue<String> animal2 = new ArrayDeque<>();
\par
// Priority Queue implementation of Queue
Queue<String> animal3 = new PriorityQueue<>();
\par
\end{lstlisting}%
\lthtmlfigureZ
\lthtmlcheckvsize\clearpage}

\stepcounter{subsection}
\stepcounter{subsection}
\stepcounter{subsubsection}
{\newpage\clearpage
\lthtmlfigureA{lstlisting1322}%
\begin{lstlisting}
import java.util.Queue;
import java.util.LinkedList;
\par
class Main {
\par
public static void main(String[] args) {
        // Creating Queue using the LinkedList class
        Queue<Integer> numbers = new LinkedList<>();
\par
// offer elements to the Queue
        numbers.offer(1);
        numbers.offer(2);
        numbers.offer(3);
        System.out.println("Queue: " + numbers);
\par
// Access elements of the Queue
        int accessedNumber = numbers.peek();
        System.out.println("Accessed Element: " + accessedNumber);
\par
// Remove elements from the Queue
        int removedNumber = numbers.poll();
        System.out.println("Removed Element: " + removedNumber);
\par
System.out.println("Updated Queue: " + numbers);
    }
}
\end{lstlisting}%
\lthtmlfigureZ
\lthtmlcheckvsize\clearpage}

{\newpage\clearpage
\lthtmlfigureA{lstlisting1325}%
\begin{lstlisting}
Queue: [1, 2, 3]
Accessed Element: 1
Removed Element: 1
Updated Queue: [2, 3]
\end{lstlisting}%
\lthtmlfigureZ
\lthtmlcheckvsize\clearpage}

{\newpage\clearpage
\lthtmlfigureA{lstlisting1332}%
\begin{lstlisting}
import java.util.PriorityQueue;
import java.util.Iterator;
\par
class Main {
    public static void main(String[] args) {
\par
// Creating a priority queue
        PriorityQueue<Integer> numbers = new PriorityQueue<>();
        numbers.add(4);
        numbers.add(2);
        numbers.add(1);
        System.out.print("PriorityQueue using iterator(): ");
\par
//Using the iterator() method
        Iterator<Integer> iterate = numbers.iterator();
        while(iterate.hasNext()) {
            System.out.print(iterate.next());
            System.out.print(", ");
        }
    }
}
\end{lstlisting}%
\lthtmlfigureZ
\lthtmlcheckvsize\clearpage}

\stepcounter{subsection}
\stepcounter{subsubsection}
{\newpage\clearpage
\lthtmlfigureA{lstlisting1344}%
\begin{lstlisting}
// Array implementation of Deque
Deque<String> animal1 = new ArrayDeque<>();
\par
// LinkedList implementation of Deque
Deque<String> animal2 = new LinkedList<>();
\end{lstlisting}%
\lthtmlfigureZ
\lthtmlcheckvsize\clearpage}

\stepcounter{subsubsection}
{\newpage\clearpage
\lthtmlfigureA{lstlisting1369}%
\begin{lstlisting}
import java.util.Deque;
import java.util.ArrayDeque;
\par
class Main {
\par
public static void main(String[] args) {
        // Creating Deque using the ArrayDeque class
        Deque<Integer> numbers = new ArrayDeque<>();
\par
// add elements to the Deque
        numbers.offer(1);
        numbers.offerLast(2);
        numbers.offerFirst(3);
        System.out.println("Deque: " + numbers);
\par
// Access elements of the Deque
        int firstElement = numbers.peekFirst();
        System.out.println("First Element: " + firstElement);
\par
int lastElement = numbers.peekLast();
        System.out.println("Last Element: " + lastElement);
\par
// Remove elements from the Deque
        int removedNumber1 = numbers.pollFirst();
        System.out.println("Removed First Element: " + removedNumber1);
\par
int removedNumber2 = numbers.pollLast();
        System.out.println("Removed Last Element: " + removedNumber2);
\par
System.out.println("Updated Deque: " + numbers);
    }
}
\end{lstlisting}%
\lthtmlfigureZ
\lthtmlcheckvsize\clearpage}

{\newpage\clearpage
\lthtmlfigureA{lstlisting1372}%
\begin{lstlisting}
Deque: [3, 1, 2]
First Element: 3
Last Element: 2
Removed First Element: 3
Removed Last Element: 2
Updated Deque: [1]
\par
\end{lstlisting}%
\lthtmlfigureZ
\lthtmlcheckvsize\clearpage}

\stepcounter{section}
\stepcounter{subsection}
{\newpage\clearpage
\lthtmlfigureA{lstlisting1382}%
\begin{lstlisting}
// Map implementation using HashMap
Map<Key, Value> numbers = new HashMap<>();
\par
\end{lstlisting}%
\lthtmlfigureZ
\lthtmlcheckvsize\clearpage}

\stepcounter{subsection}
\stepcounter{subsection}
\stepcounter{subsubsection}
{\newpage\clearpage
\lthtmlfigureA{lstlisting1413}%
\begin{lstlisting}
import java.util.Map;
import java.util.HashMap;
\par
class Main {
\par
public static void main(String[] args) {
        // Creating a map using the HashMap
        Map<String, Integer> numbers = new HashMap<>();
\par
// Insert elements to the map
        numbers.put("One", 1);
        numbers.put("Two", 2);
        System.out.println("Map: " + numbers);
\par
// Access keys of the map
        System.out.println("Keys: " + numbers.keySet());
\par
// Access values of the map
        System.out.println("Values: " + numbers.values());
\par
// Access entries of the map
        System.out.println("Entries: " + numbers.entrySet());
\par
// Remove Elements from the map
        int value = numbers.remove("Two");
        System.out.println("Removed Value: " + value);
    }
}
\end{lstlisting}%
\lthtmlfigureZ
\lthtmlcheckvsize\clearpage}

{\newpage\clearpage
\lthtmlfigureA{lstlisting1416}%
\begin{lstlisting}
Map: {One=1, Two=2}
Keys: [One, Two]
Values: [1, 2]
Entries: [One=1, Two=2]
Removed Value: 2
\end{lstlisting}%
\lthtmlfigureZ
\lthtmlcheckvsize\clearpage}

\stepcounter{subsubsection}
{\newpage\clearpage
\lthtmlfigureA{lstlisting1420}%
\begin{lstlisting}
import java.util.Map;
import java.util.TreeMap;
\par
class Main {
\par
public static void main(String[] args) {
        // Creating Map using TreeMap
        Map<String, Integer> values = new TreeMap<>();
\par
// Insert elements to map
        values.put("Second", 2);
        values.put("First", 1);
        System.out.println("Map using TreeMap: " + values);
\par
// Replacing the values
        values.replace("First", 11);
        values.replace("Second", 22);
        System.out.println("New Map: " + values);
\par
// Remove elements from the map
        int removedValue = values.remove("First");
        System.out.println("Removed Value: " + removedValue);
    }
}
\end{lstlisting}%
\lthtmlfigureZ
\lthtmlcheckvsize\clearpage}

{\newpage\clearpage
\lthtmlfigureA{lstlisting1423}%
\begin{lstlisting}
Map using TreeMap: {First=1, Second=2}
New Map: {First=11, Second=22}
Removed Value: 11
\end{lstlisting}%
\lthtmlfigureZ
\lthtmlcheckvsize\clearpage}

\stepcounter{subsection}
\stepcounter{subsubsection}
{\newpage\clearpage
\lthtmlfigureA{lstlisting1434}%
\begin{lstlisting}
import java.util.HashMap;
\par
class Main {
    public static void main(String[] args) {
        // Creating HashMap of even numbers
        HashMap<String, Integer> evenNumbers = new HashMap<>();
\par
// Using put()
        evenNumbers.put("Two", 2);
        evenNumbers.put("Four", 4);
\par
// Using putIfAbsent()
        evenNumbers.putIfAbsent("Six", 6);
        System.out.println("HashMap of even numbers: " + evenNumbers);
\par
//Creating HashMap of numbers
        HashMap<String, Integer> numbers = new HashMap<>();
        numbers.put("One", 1);
\par
// Using putAll()
        numbers.putAll(evenNumbers);
        System.out.println("HashMap of numbers: " + numbers);
    }
}
\end{lstlisting}%
\lthtmlfigureZ
\lthtmlcheckvsize\clearpage}

{\newpage\clearpage
\lthtmlfigureA{lstlisting1437}%
\begin{lstlisting}
HashMap of even numbers: {Six=6, Four=4, Two=2}
HashMap of numbers: {Six=6, One=1, Four=4, Two=2}
\end{lstlisting}%
\lthtmlfigureZ
\lthtmlcheckvsize\clearpage}

\stepcounter{subsubsection}
{\newpage\clearpage
\lthtmlfigureA{lstlisting1447}%
\begin{lstlisting}
import java.util.HashMap;
\par
class Main {
    public static void main(String[] args) {
        HashMap<String, Integer> numbers = new HashMap<>();
\par
numbers.put("One", 1);
        numbers.put("Two", 2);
        numbers.put("Three", 3);
        System.out.println("HashMap: " + numbers);
\par
// Using entrySet()
        System.out.println("Key/Value mappings: " + numbers.entrySet());
\par
// Using keySet()
        System.out.println("Keys: " + numbers.keySet());
\par
// Using values()
        System.out.println("Values: " + numbers.values());
    }
}
\end{lstlisting}%
\lthtmlfigureZ
\lthtmlcheckvsize\clearpage}

{\newpage\clearpage
\lthtmlfigureA{lstlisting1450}%
\begin{lstlisting}
HashMap: {One=1, Two=2, Three=3}
Key/Value mappings: [One=1, Two=2, Three=3]
Keys: [One, Two, Three]
Values: [1, 2, 3]
\end{lstlisting}%
\lthtmlfigureZ
\lthtmlcheckvsize\clearpage}

\stepcounter{subsubsection}
{\newpage\clearpage
\lthtmlfigureA{lstlisting1459}%
\begin{lstlisting}
import java.util.HashMap;
\par
class Main {
    public static void main(String[] args) {
\par
HashMap<String, Integer> numbers = new HashMap<>();
        numbers.put("One", 1);
        numbers.put("Two", 2);
        numbers.put("Three", 3);
        System.out.println("HashMap: " + numbers);
\par
// Using get()
        int value1 = numbers.get("Three");
        System.out.println("Returned Number: " + value1);
\par
// Using getOrDefault()
        int value2 = numbers.getOrDefault("Five", 5);
        System.out.println("Returned Number: " + value2);
    }
}
\end{lstlisting}%
\lthtmlfigureZ
\lthtmlcheckvsize\clearpage}

{\newpage\clearpage
\lthtmlfigureA{lstlisting1462}%
\begin{lstlisting}
HashMap: {One=1, Two=2, Three=3}
Returned Number: 3
Returned Number: 5
\end{lstlisting}%
\lthtmlfigureZ
\lthtmlcheckvsize\clearpage}

\stepcounter{subsubsection}
{\newpage\clearpage
\lthtmlfigureA{lstlisting1470}%
\begin{lstlisting}
import java.util.HashMap;
\par
class Main {
    public static void main(String[] args) {
\par
HashMap<String, Integer> numbers = new HashMap<>();
        numbers.put("One", 1);
        numbers.put("Two", 2);
        numbers.put("Three", 3);
        System.out.println("HashMap: " + numbers);
\par
// remove method with single parameter
        int value = numbers.remove("Two");
        System.out.println("Removed value: " + value);
\par
// remove method with two parameters
        boolean result = numbers.remove("Three", 3);
        System.out.println("Is the entry Three removed? " + result);
\par
System.out.println("Updated HashMap: " + numbers);
    }
}
\end{lstlisting}%
\lthtmlfigureZ
\lthtmlcheckvsize\clearpage}

{\newpage\clearpage
\lthtmlfigureA{lstlisting1473}%
\begin{lstlisting}
HashMap: {One=1, Two=2, Three=3}
Removed value: 2
Is the entry Three removed? True
Updated HashMap: {One=1}
\end{lstlisting}%
\lthtmlfigureZ
\lthtmlcheckvsize\clearpage}

\stepcounter{subsubsection}
{\newpage\clearpage
\lthtmlfigureA{lstlisting1483}%
\begin{lstlisting}
import java.util.HashMap;
\par
class Main {
    public static void main(String[] args) {
\par
HashMap<String, Integer> numbers = new HashMap<>();
        numbers.put("First", 1);
        numbers.put("Second", 2);
        numbers.put("Third", 3);
        System.out.println("Original HashMap: " + numbers);
\par
// Using replace()
        numbers.replace("Second", 22);
        numbers.replace("Third", 3, 33);
        System.out.println("HashMap using replace(): " + numbers);
\par
// Using replaceAll()
        numbers.replaceAll((key, oldValue) -> oldValue + 2);
        System.out.println("HashMap using replaceAll(): " + numbers);
    }
}
\end{lstlisting}%
\lthtmlfigureZ
\lthtmlcheckvsize\clearpage}

{\newpage\clearpage
\lthtmlfigureA{lstlisting1486}%
\begin{lstlisting}
Original HashMap: {Second=2, Third=3, First=1}
HashMap using replace: {Second=22, Third=33, First=1}
HashMap using replaceAll: {Second=24, Third=35, First=3}
\end{lstlisting}%
\lthtmlfigureZ
\lthtmlcheckvsize\clearpage}

\stepcounter{subsubsection}
{\newpage\clearpage
\lthtmlfigureA{lstlisting1499}%
\begin{lstlisting}
import java.util.HashMap;
\par
class Main {
    public static void main(String[] args) {
\par
HashMap<String, Integer> numbers = new HashMap<>();
        numbers.put("First", 1);
        numbers.put("Second", 2);
        System.out.println("Original HashMap: " + numbers);
\par
// Using compute()
        numbers.compute("First", (key, oldValue) -> oldValue + 2);
        numbers.compute("Second", (key, oldValue) -> oldValue + 1);
        System.out.println("HashMap using compute(): " + numbers);
\par
// Using computeIfAbsent()
        numbers.computeIfAbsent("Three", key -> 5);
        System.out.println("HashMap using computeIfAbsent(): " + numbers);
\par
// Using computeIfPresent()
        numbers.computeIfPresent("Second", (key, oldValue) -> oldValue * 2);
        System.out.println("HashMap using computeIfPresent(): " + numbers);
    }
}
\par
\end{lstlisting}%
\lthtmlfigureZ
\lthtmlcheckvsize\clearpage}

{\newpage\clearpage
\lthtmlfigureA{lstlisting1502}%
\begin{lstlisting}
Original HashMap: {Second=2, First=1}
HashMap using compute(): {Second=3, First=3}
HashMap using computeIfAbsent(): {Second=3 First=3, Three=5}
HashMap using computeIfPresent(): {Second=6, First=3, three=5}
\end{lstlisting}%
\lthtmlfigureZ
\lthtmlcheckvsize\clearpage}

\stepcounter{subsubsection}
{\newpage\clearpage
\lthtmlfigureA{lstlisting1510}%
\begin{lstlisting}
import java.util.HashMap;
\par
class Main {
    public static void main(String[] args) {
\par
HashMap<String, Integer> numbers = new HashMap<>();
        numbers.put("First", 1);
        numbers.put("Second", 2);
        System.out.println("Original HashMap: " + numbers);
\par
// Using merge() Method
        numbers.merge("First", 4, (oldValue, newValue) -> oldValue + newValue);
        System.out.println("New HashMap: " + numbers);
    }
}
\par
\end{lstlisting}%
\lthtmlfigureZ
\lthtmlcheckvsize\clearpage}

{\newpage\clearpage
\lthtmlfigureA{lstlisting1513}%
\begin{lstlisting}
Original HashMap: {Second=2, First=1}
New HashMap: {Second=2, First=5}
\end{lstlisting}%
\lthtmlfigureZ
\lthtmlcheckvsize\clearpage}

\stepcounter{subsubsection}
\stepcounter{subsection}
\stepcounter{subsubsection}
{\newpage\clearpage
\lthtmlfigureA{lstlisting1535}%
\begin{lstlisting}
import java.util.HashMap;
import java.util.Map.Entry;
\par
class Main {
    public static void main(String[] args) {
\par
// Creating a HashMap
        HashMap<String, Integer> numbers = new HashMap<>();
        numbers.put("One", 1);
        numbers.put("Two", 2);
        numbers.put("Three", 3);
        System.out.println("HashMap: " + numbers);
\par
// Accessing the key/value pair
        System.out.print("Entries: ");
        for(Entry<String, Integer> entry: numbers.entrySet()) {
            System.out.print(entry);
            System.out.print(", ");
        }
\par
// Accessing the key
        System.out.print("\nKeys: ");
        for(String key: numbers.keySet()) {
            System.out.print(key);
            System.out.print(", ");
        }
\par
// Accessing the value
        System.out.print("\nValues: ");
        for(Integer value: numbers.values()) {
            System.out.print(value);
            System.out.print(", ");
        }
    }
}
\end{lstlisting}%
\lthtmlfigureZ
\lthtmlcheckvsize\clearpage}

{\newpage\clearpage
\lthtmlfigureA{lstlisting1540}%
\begin{lstlisting}
HashMap: {One=1, Two=2, Three=3}
Entries: One=1, Two=2, Three=3
Keys: One, Two, Three,
Values: 1, 2, ,3,
\end{lstlisting}%
\lthtmlfigureZ
\lthtmlcheckvsize\clearpage}

\stepcounter{subsubsection}
{\newpage\clearpage
\lthtmlfigureA{lstlisting1548}%
\begin{lstlisting}
import java.util.HashMap;
import java.util.Iterator;
import java.util.Map.Entry;
\par
class Main {
    public static void main(String[] args) {
        // Creating a HashMap
        HashMap<String, Integer> numbers = new HashMap<>();
        numbers.put("One", 1);
        numbers.put("Two", 2);
        numbers.put("Three", 3);
        System.out.println("HashMap: " + numbers);
\par
// Creating an object of Iterator
        Iterator<Entry<String, Integer>> iterate1 = numbers.entrySet().iterator();
\par
// Accessing the Key/Value pair
        System.out.print("Entries: ");
        while(iterate1.hasNext()) {
            System.out.print(iterate1.next());
            System.out.print(", ");
        }
\par
// Accessing the key
        Iterator<String> iterate2 = numbers.keySet().iterator();
        System.out.print("\nKeys: ");
        while(iterate2.hasNext()) {
            System.out.print(iterate2.next());
            System.out.print(", ");
        }
\par
// Accessing the value
        Iterator<Integer> iterate3 = numbers.values().iterator();
         System.out.print("\nValues: ");
        while(iterate3.hasNext()) {
            System.out.print(iterate3.next());
            System.out.print(", ");
        }
    }
}
\end{lstlisting}%
\lthtmlfigureZ
\lthtmlcheckvsize\clearpage}

{\newpage\clearpage
\lthtmlfigureA{lstlisting1553}%
\begin{lstlisting}
HashMap: {One=1, Two=2, Three=3}
Entries: One=1, Two=2, Three=3
Keys: One, Two, Three,
Values: 1, 2, 3,
\end{lstlisting}%
\lthtmlfigureZ
\lthtmlcheckvsize\clearpage}

\stepcounter{section}
\stepcounter{subsection}
{\newpage\clearpage
\lthtmlfigureA{lstlisting1570}%
\begin{lstlisting}
// Set implementation using HashSet
Set<String> animals = new HashSet<>();
\end{lstlisting}%
\lthtmlfigureZ
\lthtmlcheckvsize\clearpage}

\stepcounter{subsection}
\stepcounter{subsection}
\stepcounter{subsubsection}
{\newpage\clearpage
\lthtmlfigureA{lstlisting1601}%
\begin{lstlisting}
import java.util.Set;
import java.util.HashSet;
\par
class Main {
\par
public static void main(String[] args) {
        // Creating a set using the HashSet class
        Set<Integer> set1 = new HashSet<>();
\par
// Add elements to the set1
        set1.add(2);
        set1.add(3);
        System.out.println("Set1: " + set1);
\par
// Creating another set using the HashSet class
        Set<Integer> set2 = new HashSet<>();
\par
// Add elements
        set2.add(1);
        set2.add(2);
        System.out.println("Set2: " + set2);
\par
// Union of two sets
        set2.addAll(set1);
        System.out.println("Union is: " + set2);
    }
}
\end{lstlisting}%
\lthtmlfigureZ
\lthtmlcheckvsize\clearpage}

{\newpage\clearpage
\lthtmlfigureA{lstlisting1604}%
\begin{lstlisting}
Set1: [2, 3]
Set2: [1, 2]
Union is: [1, 2, 3]
\end{lstlisting}%
\lthtmlfigureZ
\lthtmlcheckvsize\clearpage}

\stepcounter{subsubsection}
{\newpage\clearpage
\lthtmlfigureA{lstlisting1607}%
\begin{lstlisting}
import java.util.Set;
import java.util.TreeSet;
import java.util.Iterator;
\par
class Main {
\par
public static void main(String[] args) {
        // Creating a set using the TreeSet class
        Set<Integer> numbers = new TreeSet<>();
\par
// Add elements to the set
        numbers.add(2);
        numbers.add(3);
        numbers.add(1);
        System.out.println("Set using TreeSet: " + numbers);
\par
// Access Elements using iterator()
        System.out.print("Accessing elements using iterator(): ");
        Iterator<Integer> iterate = numbers.iterator();
        while(iterate.hasNext()) {
            System.out.print(iterate.next());
            System.out.print(", ");
        }
\par
}
}
\end{lstlisting}%
\lthtmlfigureZ
\lthtmlcheckvsize\clearpage}

{\newpage\clearpage
\lthtmlfigureA{lstlisting1610}%
\begin{lstlisting}
Set using TreeSet: [1, 2, 3]
Accessing elements using iterator(): 1, 2, 3,
\end{lstlisting}%
\lthtmlfigureZ
\lthtmlcheckvsize\clearpage}

\stepcounter{section}
{\newpage\clearpage
\lthtmlfigureA{lstlisting1613}%
\begin{lstlisting}
class HelloWorld {
    public static void main(String[] args) {
        System.out.println("Hello, World!"); 
    }
}
\end{lstlisting}%
\lthtmlfigureZ
\lthtmlcheckvsize\clearpage}

\stepcounter{subsection}
{\newpage\clearpage
\lthtmlfigureA{lstlisting1638}%
\begin{lstlisting}
// Creates an InputStream
InputStream object1 = new FileInputStream();
\end{lstlisting}%
\lthtmlfigureZ
\lthtmlcheckvsize\clearpage}

\stepcounter{subsection}
{\newpage\clearpage
\lthtmlfigureA{lstlisting1659}%
\begin{lstlisting}
This is a line of text inside the file.
\end{lstlisting}%
\lthtmlfigureZ
\lthtmlcheckvsize\clearpage}

{\newpage\clearpage
\lthtmlfigureA{lstlisting1663}%
\begin{lstlisting}
import java.io.FileInputStream;
import java.io.InputStream;
\par
public class Main {
    public static void main(String args[]) {
\par
byte[] array = new byte[100];
\par
try {
            InputStream input = new FileInputStream("input.txt");
\par
System.out.println("Available bytes in the file: " + input.available());
\par
// Read byte from the input stream
            input.read(array);
            System.out.println("Data read from the file: ");
\par
// Convert byte array into string
            String data = new String(array);
            System.out.println(data);
\par
// Close the input stream
            input.close();
        }
        catch (Exception e) {
            e.getStackTrace();
        }
    }
}
\end{lstlisting}%
\lthtmlfigureZ
\lthtmlcheckvsize\clearpage}

{\newpage\clearpage
\lthtmlfigureA{lstlisting1667}%
\begin{lstlisting}
Available bytes in the file: 35
Data read from the file:
This is a line of text inside the file
\end{lstlisting}%
\lthtmlfigureZ
\lthtmlcheckvsize\clearpage}

\stepcounter{subsection}
{\newpage\clearpage
\lthtmlfigureA{lstlisting1682}%
\begin{lstlisting}
// Creates an OutputStream
OutputStream object = new FileOutputStream();
\end{lstlisting}%
\lthtmlfigureZ
\lthtmlcheckvsize\clearpage}

\stepcounter{subsection}
{\newpage\clearpage
\lthtmlfigureA{lstlisting1698}%
\begin{lstlisting}
import java.io.FileOutputStream;
import java.io.OutputStream;
\par
public class Main {
\par
public static void main(String args[]) {
        String data = "This is a line of text inside the file.";
\par
try {
            OutputStream out = new FileOutputStream("output.txt");
\par
// Converts the string into bytes
            byte[] dataBytes = data.getBytes();
\par
// Writes data to the output stream
            out.write(dataBytes);
            System.out.println("Data is written to the file.");
\par
// Closes the output stream
            out.close();
        }
\par
catch (Exception e) {
            e.getStackTrace();
        }
    }
}
\end{lstlisting}%
\lthtmlfigureZ
\lthtmlcheckvsize\clearpage}

\stepcounter{section}
{\newpage\clearpage
\lthtmlfigureA{lstlisting1716}%
\begin{lstlisting}
// Creates a Reader
Reader input = new FileReader();
\end{lstlisting}%
\lthtmlfigureZ
\lthtmlcheckvsize\clearpage}

\stepcounter{subsection}
{\newpage\clearpage
\lthtmlfigureA{lstlisting1737}%
\begin{lstlisting}
import java.io.Reader;
import java.io.FileReader;
\par
class Main {
    public static void main(String[] args) {
\par
// Creates an array of character
        char[] array = new char[100];
\par
try {
            // Creates a reader using the FileReader
            Reader input = new FileReader("input.txt");
\par
// Checks if reader is ready 
            System.out.println("Is there data in the stream?  " + input.ready());
\par
// Reads characters
            input.read(array);
            System.out.println("Data in the stream:");
            System.out.println(array);
\par
// Closes the reader
            input.close();
        }
\par
catch(Exception e) {
            e.getStackTrace();
        }
    }
}
\end{lstlisting}%
\lthtmlfigureZ
\lthtmlcheckvsize\clearpage}

{\newpage\clearpage
\lthtmlfigureA{lstlisting1741}%
\begin{lstlisting}
Is there data in the stream?  true
Data in the stream:
This is a line of text inside the file.
\end{lstlisting}%
\lthtmlfigureZ
\lthtmlcheckvsize\clearpage}

\stepcounter{subsection}
{\newpage\clearpage
\lthtmlfigureA{lstlisting1761}%
\begin{lstlisting}
// Creates a Writer
Writer output = new FileWriter();
\end{lstlisting}%
\lthtmlfigureZ
\lthtmlcheckvsize\clearpage}

\stepcounter{subsection}
{\newpage\clearpage
\lthtmlfigureA{lstlisting1777}%
\begin{lstlisting}
import java.io.FileWriter;
import java.io.Writer;
\par
public class Main {
\par
public static void main(String args[]) {
\par
String data = "This is the data in the output file";
\par
try {
            // Creates a Writer using FileWriter
            Writer output = new FileWriter("output.txt");
\par
// Writes string to the file
            output.write(data);
\par
// Closes the writer
            output.close();
        }
\par
catch (Exception e) {
            e.getStackTrace();
        }
    }
}
\par
\end{lstlisting}%
\lthtmlfigureZ
\lthtmlcheckvsize\clearpage}

\stepcounter{chapter}
\stepcounter{section}
\stepcounter{subsection}
{\newpage\clearpage
\lthtmlfigureA{lstlisting1794}%
\begin{lstlisting}
if (expression) {
    // statements
}
\end{lstlisting}%
\lthtmlfigureZ
\lthtmlcheckvsize\clearpage}

{\newpage\clearpage
\lthtmlfigureA{lstlisting1803}%
\begin{lstlisting}
class IfStatement {
    public static void main(String[] args) {
\par
int number = 10;
\par
// checks if number is greater than 0
        if (number > 0) {
            System.out.println("The number is positive.");
        }
        System.out.println("This statement is always executed.");
    }
}
\end{lstlisting}%
\lthtmlfigureZ
\lthtmlcheckvsize\clearpage}

{\newpage\clearpage
\lthtmlfigureA{lstlisting1806}%
\begin{lstlisting}
The number is positive.
This statement is always executed.
\end{lstlisting}%
\lthtmlfigureZ
\lthtmlcheckvsize\clearpage}

{\newpage\clearpage
\lthtmlfigureA{lstlisting1818}%
\begin{lstlisting}
class Main {
  public static void main(String[] args) {
    // create a string variable
    String language = "Java";
\par
// if statement
    if(language == "Java") {
      System.out.println("This is best programming language.");
    } 
  }
}
\end{lstlisting}%
\lthtmlfigureZ
\lthtmlcheckvsize\clearpage}

{\newpage\clearpage
\lthtmlfigureA{lstlisting1828}%
\begin{lstlisting}
if (expression) {
   // codes
}
else {
  // some other code
}
\end{lstlisting}%
\lthtmlfigureZ
\lthtmlcheckvsize\clearpage}

{\newpage\clearpage
\lthtmlfigureA{lstlisting1834}%
\begin{lstlisting}
class IfElse {
    public static void main(String[] args) {    	
        int number = 10;
\par
// checks if number is greater than 0	 
        if (number > 0) {
            System.out.println("The number is positive.");
        }
        else {
            System.out.println("The number is not positive.");
        }
\par
System.out.println("This statement is always executed.");
    }
}
\end{lstlisting}%
\lthtmlfigureZ
\lthtmlcheckvsize\clearpage}

{\newpage\clearpage
\lthtmlfigureA{lstlisting1848}%
\begin{lstlisting}
The number is not positive.
This statement is always executed.
\end{lstlisting}%
\lthtmlfigureZ
\lthtmlcheckvsize\clearpage}

{\newpage\clearpage
\lthtmlfigureA{lstlisting1854}%
\begin{lstlisting}
if (expression1) {
   // codes
}
else if(expression2) {
   // codes
}
else if (expression3) {
   // codes
}
.
.
else {
   // codes
}
\end{lstlisting}%
\lthtmlfigureZ
\lthtmlcheckvsize\clearpage}

{\newpage\clearpage
\lthtmlfigureA{lstlisting1866}%
\begin{lstlisting}
class Ladder {
    public static void main(String[] args) {   
\par
int number = 0;
\par
// checks if number is greater than 0	 
        if (number > 0) {
            System.out.println("The number is positive.");
        }
\par
// checks if number is less than 0
        else if (number < 0) {
            System.out.println("The number is negative.");
        }
        else {
            System.out.println("The number is 0.");
        } 
    }
}
\end{lstlisting}%
\lthtmlfigureZ
\lthtmlcheckvsize\clearpage}

\stepcounter{subsection}
{\newpage\clearpage
\lthtmlfigureA{lstlisting1883}%
\begin{lstlisting}
class Number {
    public static void main(String[] args) {
\par
// declaring double type variables
        Double n1 = -1.0, n2 = 4.5, n3 = -5.3, largestNumber;
\par
// checks if n1 is greater than or equal to n2
        if (n1 >= n2) {
\par
// if...else statement inside the if block
            // checks if n1 is greater than or equal to n3
            if (n1 >= n3) {
                largestNumber = n1;
            }
\par
else {
                largestNumber = n3;
            }
        }
        else {
\par
// if..else statement inside else block
            // checks if n2 is greater than or equal to n3
            if (n2 >= n3) {
                largestNumber = n2;
            }
\par
else {
                largestNumber = n3;
            }
        }
\par
System.out.println("The largest number is " + largestNumber);
    }
}
\end{lstlisting}%
\lthtmlfigureZ
\lthtmlcheckvsize\clearpage}

\stepcounter{section}
{\newpage\clearpage
\lthtmlfigureA{lstlisting1891}%
\begin{lstlisting}
if (expression) {
   number = 10;
}
else {
   number = -10;
}
\end{lstlisting}%
\lthtmlfigureZ
\lthtmlcheckvsize\clearpage}

{\newpage\clearpage
\lthtmlfigureA{lstlisting1895}%
\begin{lstlisting}
number = (expression) ? expressionTrue : expressinFalse;
\end{lstlisting}%
\lthtmlfigureZ
\lthtmlcheckvsize\clearpage}

{\newpage\clearpage
\lthtmlfigureA{lstlisting1906}%
\begin{lstlisting}
class Operator {
   public static void main(String[] args) {   
\par
Double number = -5.5;
      String result;
\par
result = (number > 0.0) ? "positive" : "not positive";
      System.out.println(number + " is " + result);
   }
}
\end{lstlisting}%
\lthtmlfigureZ
\lthtmlcheckvsize\clearpage}

\stepcounter{subsubsection}
{\newpage\clearpage
\lthtmlfigureA{lstlisting1911}%
\begin{lstlisting}
if (expression1) {
	result = 1;
} else if (expression2) {
	result = 2;
} else if (expression3) {
	result = 3;
} else {
	result = 0;
}
\end{lstlisting}%
\lthtmlfigureZ
\lthtmlcheckvsize\clearpage}

{\newpage\clearpage
\lthtmlfigureA{lstlisting1917}%
\begin{lstlisting}
result = (expression1) ? 1 : (expression2) ? 2 : (expression3) ? 3 : 0;
\end{lstlisting}%
\lthtmlfigureZ
\lthtmlcheckvsize\clearpage}

\stepcounter{section}
{\newpage\clearpage
\lthtmlfigureA{lstlisting1923}%
\begin{lstlisting}
switch (variable/expression) {
case value1:
   // statements of case1
   break;
\par
case value2:
   // statements of case2
   break;
\par
.. .. ...
   .. .. ...
\par
default:
   // default statements
}
\end{lstlisting}%
\lthtmlfigureZ
\lthtmlcheckvsize\clearpage}

{\newpage\clearpage
\lthtmlfigureA{lstlisting1944}%
\begin{lstlisting}
class Main {
    public static void main(String[] args) {
\par
int week = 4;
        String day;
\par
// switch statement to check day
        switch (week) {
            case 1:
                day = "Sunday";
                break;
            case 2:
                day = "Monday";
                break;
            case 3:
                day = "Tuesday";
                break;
\par
// match the value of week
            case 4:
                day = "Wednesday";
                break;
            case 5:
                day = "Thursday";
                break;
            case 6:
                day = "Friday";
                break;
            case 7:
                day = "Saturday";
                break;
            default:
                day = "Invalid day";
                break;
        }
        System.out.println("The day is " + day);
    }
}
\end{lstlisting}%
\lthtmlfigureZ
\lthtmlcheckvsize\clearpage}

{\newpage\clearpage
\lthtmlfigureA{lstlisting1958}%
\begin{lstlisting}
import java.util.Scanner;
\par
class Main {
    public static void main(String[] args) {
\par
char operator;
        Double number1, number2, result;
\par
// create an object of Scanner class
        Scanner scanner = new Scanner(System.in);
        System.out.print("Enter operator (either +, -, * or /): ");
\par
// ask user to enter operator
        operator = scanner.next().charAt(0);
        System.out.print("Enter number1 and number2 respectively: ");
\par
// ask user to enter numbers
        number1 = scanner.nextDouble();
        number2 = scanner.nextDouble();
\par
switch (operator) {
\par
// performs addition between numbers
            case '+':
                result = number1 + number2;
                System.out.print(number1 + "+" + number2 + " = " + result);
                break;
\par
// performs subtraction between numbers
            case '-':
                result = number1 - number2;
                System.out.print(number1 + "-" + number2 + " = " + result);
                break;
\par
// performs multiplication between numbers
            case '*':
                result = number1 * number2;
                System.out.print(number1 + "*" + number2 + " = " + result);
                break;
\par
// performs division between numbers
            case '/':
                result = number1 / number2;
                System.out.print(number1 + "/" + number2 + " = " + result);
                break;
\par
default:
                System.out.println("Invalid operator!");
                break;
        }
    }
}
\end{lstlisting}%
\lthtmlfigureZ
\lthtmlcheckvsize\clearpage}

{\newpage\clearpage
\lthtmlfigureA{lstlisting1961}%
\begin{lstlisting}
Enter operator (either +, -, * or /): *
Enter number1 and number2 respectively: 1.4
-5.3
1.4*-5.3 = -7.419999999999999
\end{lstlisting}%
\lthtmlfigureZ
\lthtmlcheckvsize\clearpage}

\stepcounter{section}
{\newpage\clearpage
\lthtmlfigureA{lstlisting1965}%
\begin{lstlisting}
for (initialization; testExpression; update)
{
    // codes inside for loop's body
}
\end{lstlisting}%
\lthtmlfigureZ
\lthtmlcheckvsize\clearpage}

{\newpage\clearpage
\lthtmlfigureA{lstlisting1974}%
\begin{lstlisting}
// Program to print a sentence 10 times
\par
class Loop {
    public static void main(String[] args) {
\par
for (int i = 1; i <= 10; ++i) {
            System.out.println("Line " + i);
        }
    }
}
\end{lstlisting}%
\lthtmlfigureZ
\lthtmlcheckvsize\clearpage}

{\newpage\clearpage
\lthtmlfigureA{lstlisting1977}%
\begin{lstlisting}
Line 1
Line 2
Line 3
Line 4
Line 5
Line 6
Line 7
Line 8
Line 9
Line 10
\end{lstlisting}%
\lthtmlfigureZ
\lthtmlcheckvsize\clearpage}

{\newpage\clearpage
\lthtmlfigureA{lstlisting1991}%
\begin{lstlisting}
// Program to find the sum of natural numbers from 1 to 1000.
\par
class Number {
    public static void main(String[] args) {
\par
int sum = 0;
\par
for (int i = 1; i <= 1000; ++i) {
            sum += i;     // sum = sum + i
        }
\par
System.out.println("Sum = " + sum);
    }
}
\end{lstlisting}%
\lthtmlfigureZ
\lthtmlcheckvsize\clearpage}

\stepcounter{subsubsection}
{\newpage\clearpage
\lthtmlfigureA{lstlisting1998}%
\begin{lstlisting}
// Infinite for Loop
\par
class Infinite {
    public static void main(String[] args) {
\par
int sum = 0;
\par
for (int i = 1; i <= 10; --i) {
            System.out.println("Hello");
        }
    }
}
\end{lstlisting}%
\lthtmlfigureZ
\lthtmlcheckvsize\clearpage}

\stepcounter{subsection}
{\newpage\clearpage
\lthtmlfigureA{lstlisting2006}%
\begin{lstlisting}
for (int a : array) {
    System.out.println(a);
}
\end{lstlisting}%
\lthtmlfigureZ
\lthtmlcheckvsize\clearpage}

{\newpage\clearpage
\lthtmlfigureA{lstlisting2012}%
\begin{lstlisting}
for(data_type item : collections) {
    ...
}
\end{lstlisting}%
\lthtmlfigureZ
\lthtmlcheckvsize\clearpage}

{\newpage\clearpage
\lthtmlfigureA{lstlisting2017}%
\begin{lstlisting}
// The program below calculates the sum of all elements of an integer array.
\par
class Main {
 public static void main(String[] args) {
\par
// create array
   int[] numbers = {3, 4, 5, -5, 0, 12};
   int sum = 0;
\par
// for each loop 
   for (int number: numbers) {
     sum += number;
   }
\par
System.out.println("Sum = " + sum);
 }
}
\end{lstlisting}%
\lthtmlfigureZ
\lthtmlcheckvsize\clearpage}

\stepcounter{subsubsection}
{\newpage\clearpage
\lthtmlfigureA{lstlisting2025}%
\begin{lstlisting}
class Main {
 public static void main(String[] args) {
\par
char[] vowels = {'a', 'e', 'i', 'o', 'u'};
\par
// using for loop
   for (int i = 0; i < vowels.length; ++ i) {
     System.out.println(vowels[i]);
   }
 }
}
\end{lstlisting}%
\lthtmlfigureZ
\lthtmlcheckvsize\clearpage}

{\newpage\clearpage
\lthtmlfigureA{lstlisting2029}%
\begin{lstlisting}
a
e
i
o
u
\end{lstlisting}%
\lthtmlfigureZ
\lthtmlcheckvsize\clearpage}

\stepcounter{subsubsection}
{\newpage\clearpage
\lthtmlfigureA{lstlisting2032}%
\begin{lstlisting}
class Main {
 public static void main(String[] args) {
\par
// create a char array  
   char[] vowels = {'a', 'e', 'i', 'o', 'u'};
\par
// foreach loop
   for (char item: vowels) {
     System.out.println(item);
   }
 }
}
\end{lstlisting}%
\lthtmlfigureZ
\lthtmlcheckvsize\clearpage}

\stepcounter{subsection}
\stepcounter{section}
{\newpage\clearpage
\lthtmlfigureA{lstlisting2049}%
\begin{lstlisting}
while (testExpression) {
    // codes inside the body of while loop
}
\end{lstlisting}%
\lthtmlfigureZ
\lthtmlcheckvsize\clearpage}

\stepcounter{subsubsection}
{\newpage\clearpage
\lthtmlfigureA{lstlisting2058}%
\begin{lstlisting}
// Program to print line 10 times
\par
class Loop {
    public static void main(String[] args) {
\par
int i = 1;
\par
while (i <= 10) {
            System.out.println("Line " + i);
            ++i;
        }
    }
}
\end{lstlisting}%
\lthtmlfigureZ
\lthtmlcheckvsize\clearpage}

{\newpage\clearpage
\lthtmlfigureA{lstlisting2074}%
\begin{lstlisting}
// Program to find the sum of natural numbers from 1 to 100.
\par
class AssignmentOperator {
    public static void main(String[] args) {
\par
int sum = 0, i = 100;
\par
while (i != 0) {
            sum += i;     // sum = sum + i;
            --i;
        }
\par
System.out.println("Sum = " + sum);
    }
}
\end{lstlisting}%
\lthtmlfigureZ
\lthtmlcheckvsize\clearpage}

\stepcounter{subsection}
{\newpage\clearpage
\lthtmlfigureA{lstlisting2085}%
\begin{lstlisting}
do {
   // codes inside body of do while loop
} while (testExpression);
\end{lstlisting}%
\lthtmlfigureZ
\lthtmlcheckvsize\clearpage}

{\newpage\clearpage
\lthtmlfigureA{lstlisting2096}%
\begin{lstlisting}
import java.util.Scanner;
\par
class Sum {
    public static void main(String[] args) {
\par
Double number, sum = 0.0;
        // creates an object of Scanner class
        Scanner input = new Scanner(System.in);
\par
do {
\par
// takes input from the user
            System.out.print("Enter a number: ");
            number = input.nextDouble();
            sum += number;
        } while (number != 0.0);  // test expression
\par
System.out.println("Sum = " + sum);
    }
}
\end{lstlisting}%
\lthtmlfigureZ
\lthtmlcheckvsize\clearpage}

{\newpage\clearpage
\lthtmlfigureA{lstlisting2099}%
\begin{lstlisting}
Enter a number: 2.5
Enter a number: 23.3
Enter a number: -4.2
Enter a number: 3.4
Enter a number: 0
Sum = 25.0
\end{lstlisting}%
\lthtmlfigureZ
\lthtmlcheckvsize\clearpage}

\stepcounter{subsubsection}
{\newpage\clearpage
\lthtmlfigureA{lstlisting2106}%
\begin{lstlisting}
// Infinite while loop
while (true) {
   // body of while loop
}
\end{lstlisting}%
\lthtmlfigureZ
\lthtmlcheckvsize\clearpage}

{\newpage\clearpage
\lthtmlfigureA{lstlisting2109}%
\begin{lstlisting}
// Infinite while loop
int i = 100;
while (i == 100) {
   System.out.print("Hey!");
}
\end{lstlisting}%
\lthtmlfigureZ
\lthtmlcheckvsize\clearpage}

\stepcounter{section}
{\newpage\clearpage
\lthtmlfigureA{lstlisting2114}%
\begin{lstlisting}
class Test {
    public static void main(String[] args) {
\par
// for loop
        for (int i = 1; i <= 10; ++i) {
\par
// if the value of i is 5 the loop terminates  
            if (i == 5) {
                break;
            }      
            System.out.println(i);
        }   
    }
}
\end{lstlisting}%
\lthtmlfigureZ
\lthtmlcheckvsize\clearpage}

{\newpage\clearpage
\lthtmlfigureA{lstlisting2117}%
\begin{lstlisting}
1
2
3
4
\end{lstlisting}%
\lthtmlfigureZ
\lthtmlcheckvsize\clearpage}

{\newpage\clearpage
\lthtmlfigureA{lstlisting2122}%
\begin{lstlisting}
if (i == 5) {
    break;
}
\end{lstlisting}%
\lthtmlfigureZ
\lthtmlcheckvsize\clearpage}

{\newpage\clearpage
\lthtmlfigureA{lstlisting2128}%
\begin{lstlisting}
import java.util.Scanner;
\par
class UserInputSum {
    public static void main(String[] args) {
\par
Double number, sum = 0.0;
\par
// create an object of Scanner
        Scanner input = new Scanner(System.in);
\par
while (true) {
            System.out.print("Enter a number: ");
\par
// takes double input from user
            number = input.nextDouble();
\par
// if number is negative the loop terminates
            if (number < 0.0) {
                break;
            }
\par
sum += number;
        }
        System.out.println("Sum = " + sum);
    }
}
\end{lstlisting}%
\lthtmlfigureZ
\lthtmlcheckvsize\clearpage}

{\newpage\clearpage
\lthtmlfigureA{lstlisting2131}%
\begin{lstlisting}
Enter a number: 3.2
Enter a number: 5
Enter a number: 2.3
Enter a number: 0
Enter a number: -4.5
Sum = 10.5
\end{lstlisting}%
\lthtmlfigureZ
\lthtmlcheckvsize\clearpage}

{\newpage\clearpage
\lthtmlfigureA{lstlisting2135}%
\begin{lstlisting}
if (number < 0.0) {
    break;
}
\end{lstlisting}%
\lthtmlfigureZ
\lthtmlcheckvsize\clearpage}

\stepcounter{subsubsection}
{\newpage\clearpage
\lthtmlfigureA{lstlisting2142}%
\begin{lstlisting}
while (testExpression) {
   // codes
   second:
   while (testExpression) {
      // codes
      while(testExpression) {
         // codes
         break second;
      }
   }
   // control jumps here
}
\end{lstlisting}%
\lthtmlfigureZ
\lthtmlcheckvsize\clearpage}

{\newpage\clearpage
\lthtmlfigureA{lstlisting2149}%
\begin{lstlisting}
class LabeledBreak {
    public static void main(String[] args) {
\par
// the for loop is labeled as first   
        first:
        for( int i = 1; i < 5; i++) {
\par
// the for loop is labeled as second
            second:
            for(int j = 1; j < 3; j ++ ) {
                System.out.println("i = " + i + "; j = " +j);
\par
// the break statement breaks the first for loop
                if ( i == 2)
                    break first;
            }
        }
    }
}
\end{lstlisting}%
\lthtmlfigureZ
\lthtmlcheckvsize\clearpage}

{\newpage\clearpage
\lthtmlfigureA{lstlisting2152}%
\begin{lstlisting}
i = 1; j = 1
i = 1; j = 2
i = 2; j = 1
\end{lstlisting}%
\lthtmlfigureZ
\lthtmlcheckvsize\clearpage}

{\newpage\clearpage
\lthtmlfigureA{lstlisting2155}%
\begin{lstlisting}
first:
for(int i = 1; i < 5; i++) {...}
\end{lstlisting}%
\lthtmlfigureZ
\lthtmlcheckvsize\clearpage}

{\newpage\clearpage
\lthtmlfigureA{lstlisting2161}%
\begin{lstlisting}
class LabeledBreak {
    public static void main(String[] args) {
\par
// the for loop is labeled as first
        first:
        for( int i = 1; i < 5; i++) {
\par
// the for loop is labeled as second
            second:
            for(int j = 1; j < 3; j ++ ) {
\par
System.out.println("i = " + i + "; j = " +j);
\par
// the break statement terminates the loop labeled as second   
                if ( i == 2)
                    break second;
            }
        }
    }
}
\end{lstlisting}%
\lthtmlfigureZ
\lthtmlcheckvsize\clearpage}

{\newpage\clearpage
\lthtmlfigureA{lstlisting2164}%
\begin{lstlisting}
i = 1; j = 1
i = 1; j = 2
i = 2; j = 1
i = 3; j = 1
i = 3; j = 2
i = 4; j = 1
i = 4; j = 2
\end{lstlisting}%
\lthtmlfigureZ
\lthtmlcheckvsize\clearpage}

\stepcounter{section}
{\newpage\clearpage
\lthtmlfigureA{lstlisting2171}%
\begin{lstlisting}
class Test {
    public static void main(String[] args) {
\par
// for loop
        for (int i = 1; i <= 10; ++i) {
\par
// if value of i is between 4 and 9, continue is executed 
            if (i > 4 && i < 9) {
                continue;
            }      
            System.out.println(i);
        }   
    }
}
\end{lstlisting}%
\lthtmlfigureZ
\lthtmlcheckvsize\clearpage}

{\newpage\clearpage
\lthtmlfigureA{lstlisting2174}%
\begin{lstlisting}
1
2
​​​​3
4
9
10
\end{lstlisting}%
\lthtmlfigureZ
\lthtmlcheckvsize\clearpage}

{\newpage\clearpage
\lthtmlfigureA{lstlisting2179}%
\begin{lstlisting}
if (i > 5 && i < 9) {
    continue;
}
\end{lstlisting}%
\lthtmlfigureZ
\lthtmlcheckvsize\clearpage}

{\newpage\clearpage
\lthtmlfigureA{lstlisting2186}%
\begin{lstlisting}
import java.util.Scanner;
\par
class AssignmentOperator {
    public static void main(String[] args) {
\par
Double number, sum = 0.0;
        // create an object of Scanner
        Scanner input = new Scanner(System.in);
\par
for (int i = 1; i < 6; ++i) {
            System.out.print("Enter a number: ");
            // takes double type input from the user
            number = input.nextDouble();
\par
// if number is negative, the iteration is skipped
            if (number <= 0.0) {
                continue;
            }
\par
sum += number;
        }
        System.out.println("Sum = " + sum);
    }
}
\end{lstlisting}%
\lthtmlfigureZ
\lthtmlcheckvsize\clearpage}

{\newpage\clearpage
\lthtmlfigureA{lstlisting2189}%
\begin{lstlisting}
Enter a number: 2.2
Enter a number: 5.6
Enter a number: 0
Enter a number: -2.4
Enter a number: -3
Sum = 7.8
\end{lstlisting}%
\lthtmlfigureZ
\lthtmlcheckvsize\clearpage}

\stepcounter{subsubsection}
{\newpage\clearpage
\lthtmlfigureA{lstlisting2198}%
\begin{lstlisting}
class LabeledContinue {
    public static void main(String[] args) {
\par
// the outer for loop is labeled as label      
        first:
        for (int i = 1; i < 6; ++i) {
            for (int j = 1; j < 5; ++j) {
                if (i == 3 || j == 2)
\par
// skips the iteration of label (outer for loop)
                    continue first;
                System.out.println("i = " + i + "; j = " + j); 
            }
        } 
    }
}
\end{lstlisting}%
\lthtmlfigureZ
\lthtmlcheckvsize\clearpage}

{\newpage\clearpage
\lthtmlfigureA{lstlisting2201}%
\begin{lstlisting}
i = 1; j = 1
i = 2; j = 1
i = 4; j = 1
i = 5; j = 1
\end{lstlisting}%
\lthtmlfigureZ
\lthtmlcheckvsize\clearpage}

{\newpage\clearpage
\lthtmlfigureA{lstlisting2205}%
\begin{lstlisting}
if (i==3 || j==2)
    continue first;
\end{lstlisting}%
\lthtmlfigureZ
\lthtmlcheckvsize\clearpage}

{\newpage\clearpage
\lthtmlfigureA{lstlisting2209}%
\begin{lstlisting}
first:
for (int i = 1; i < 6; ++i) {..}
\end{lstlisting}%
\lthtmlfigureZ
\lthtmlcheckvsize\clearpage}

\stepcounter{chapter}
\stepcounter{section}
\stepcounter{section}
\stepcounter{section}
\stepcounter{subsubsection}
\stepcounter{subsubsection}
\stepcounter{subsubsection}
\stepcounter{subsubsection}
\stepcounter{subsubsection}
\stepcounter{subsubsection}
\stepcounter{subsubsection}
{\newpage\clearpage
\lthtmlinlinemathA{tex2html_wrap_inline3010}%
$\lstinline{this}$%
\lthtmlindisplaymathZ
\lthtmlcheckvsize\clearpage}

{\newpage\clearpage
\lthtmlinlinemathA{tex2html_wrap_inline3012}%
$\lstinline{self}$%
\lthtmlindisplaymathZ
\lthtmlcheckvsize\clearpage}

\stepcounter{section}
\stepcounter{section}
\stepcounter{subsection}
\stepcounter{subsection}
\stepcounter{subsection}
\stepcounter{subsection}
\stepcounter{subsection}
\stepcounter{subsection}
\stepcounter{section}
{\newpage\clearpage
\lthtmlfigureA{lstlisting2301}%
\begin{lstlisting}
enum Size {
    constant1, constant2, …, constantN;
\par
// methods and fields	
}
\end{lstlisting}%
\lthtmlfigureZ
\lthtmlcheckvsize\clearpage}

{\newpage\clearpage
\lthtmlfigureA{lstlisting2310}%
\begin{lstlisting}
enum Size { 
   SMALL, MEDIUM, LARGE, EXTRALARGE 
}
\end{lstlisting}%
\lthtmlfigureZ
\lthtmlcheckvsize\clearpage}

{\newpage\clearpage
\lthtmlfigureA{lstlisting2316}%
\begin{lstlisting}
enum Size {
   SMALL, MEDIUM, LARGE, EXTRALARGE
}
\par
class Main {
   public static void main(String[] args) {
      System.out.println(Size.SMALL);
      System.out.println(Size.MEDIUM);
   }
}
\end{lstlisting}%
\lthtmlfigureZ
\lthtmlcheckvsize\clearpage}

{\newpage\clearpage
\lthtmlfigureA{lstlisting2320}%
\begin{lstlisting}
SMALL
MEDIUM
\end{lstlisting}%
\lthtmlfigureZ
\lthtmlcheckvsize\clearpage}

{\newpage\clearpage
\lthtmlfigureA{lstlisting2324}%
\begin{lstlisting}
pizzaSize = Size.SMALL;
pizzaSize = Size.MEDIUM;
pizzaSize = Size.LARGE;
pizzaSize = Size.EXTRALARGE;
\end{lstlisting}%
\lthtmlfigureZ
\lthtmlcheckvsize\clearpage}

{\newpage\clearpage
\lthtmlfigureA{lstlisting2326}%
\begin{lstlisting}
enum Size {
 SMALL, MEDIUM, LARGE, EXTRALARGE
}
\par
class Test {
 Size pizzaSize;
 public Test(Size pizzaSize) {
   this.pizzaSize = pizzaSize;
 }
 public void orderPizza() {
   switch(pizzaSize) {
     case SMALL:
       System.out.println("I ordered a small size pizza.");
       break;
     case MEDIUM:
       System.out.println("I ordered a medium size pizza.");
       break;
     default:
       System.out.println("I don't know which one to order.");
       break;
   }
 }
}
\par
class Main {
 public static void main(String[] args) {
   Test t1 = new Test(Size.MEDIUM);
   t1.orderPizza();
 }
}
\end{lstlisting}%
\lthtmlfigureZ
\lthtmlcheckvsize\clearpage}

{\newpage\clearpage
\lthtmlfigureA{lstlisting2332}%
\begin{lstlisting}
I ordered a medium size pizza.
\end{lstlisting}%
\lthtmlfigureZ
\lthtmlcheckvsize\clearpage}

{\newpage\clearpage
\lthtmlfigureA{lstlisting2342}%
\begin{lstlisting}
enum Size{
   SMALL, MEDIUM, LARGE, EXTRALARGE;
\par
public String getSize() {
\par
// this will refer to the object SMALL
      switch(this) {
         case SMALL:
          return "small";
\par
case MEDIUM:
          return "medium";
\par
case LARGE:
          return "large";
\par
case EXTRALARGE:
          return "extra large";
\par
default:
          return null;
      }
   }
\par
public static void main(String[] args) {
\par
// calling the method getSize() using the object SMALL
      System.out.println("The size of the pizza is " + Size.SMALL.getSize());
   }
}
\end{lstlisting}%
\lthtmlfigureZ
\lthtmlcheckvsize\clearpage}

\stepcounter{subsection}
\stepcounter{subsection}
{\newpage\clearpage
\lthtmlfigureA{lstlisting2390}%
\begin{lstlisting}
class Size {
   public final static int SMALL = 1;
   public final static int MEDIUM = 2;
   public final static int LARGE = 3;
   public final static int EXTRALARGE = 4;
}
\end{lstlisting}%
\lthtmlfigureZ
\lthtmlcheckvsize\clearpage}

\stepcounter{subsection}
{\newpage\clearpage
\lthtmlfigureA{lstlisting2405}%
\begin{lstlisting}
enum Size {
\par
// enum constants calling the enum constructors 
   SMALL("The size is small."),
   MEDIUM("The size is medium."),
   LARGE("The size is large."),
   EXTRALARGE("The size is extra large.");
\par
private final String pizzaSize;
\par
// private enum constructor
   private Size(String pizzaSize) {
      this.pizzaSize = pizzaSize;
   }
\par
public String getSize() {
      return pizzaSize;
   }
}
\par
class Main {
   public static void main(String[] args) {
      Size size = Size.SMALL;
      System.out.println(size.getSize());
   }
}
\end{lstlisting}%
\lthtmlfigureZ
\lthtmlcheckvsize\clearpage}

\stepcounter{subsection}
{\newpage\clearpage
\lthtmlfigureA{lstlisting2428}%
\begin{lstlisting}
enum Size {
   SMALL, MEDIUM, LARGE, EXTRALARGE
}
\par
class Main {
   public static void main(String[] args) {
\par
System.out.println("string value of SMALL is " + Size.SMALL.toString());
      System.out.println("string value of MEDIUM is " + Size.MEDIUM.name());
\par
}
}
\end{lstlisting}%
\lthtmlfigureZ
\lthtmlcheckvsize\clearpage}

{\newpage\clearpage
\lthtmlfigureA{lstlisting2432}%
\begin{lstlisting}
string value of SMALL is SMALL
string value of MEDIUM is MEDIUM
\end{lstlisting}%
\lthtmlfigureZ
\lthtmlcheckvsize\clearpage}

{\newpage\clearpage
\lthtmlfigureA{lstlisting2437}%
\begin{lstlisting}
enum Size {
   SMALL {
\par
// overriding toString() for SMALL
      public String toString() {
        return "The size is small.";
      }
   },
\par
MEDIUM {
\par
// overriding toString() for MEDIUM
      public String toString() {
        return "The size is medium.";
      }
   };
}
\par
class Main {
   public static void main(String[] args) {
      System.out.println(Size.MEDIUM.toString());
   }
}
\end{lstlisting}%
\lthtmlfigureZ
\lthtmlcheckvsize\clearpage}

\stepcounter{chapter}

\end{document}
